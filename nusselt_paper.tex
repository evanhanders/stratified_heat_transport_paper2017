%\documentclass[iop]{emulateapj}
\documentclass[twocolumn]{aastex6}
\usepackage{amsmath}
\usepackage{graphicx}
\usepackage{epstopdf}
\usepackage{listings}
\usepackage{color}
\usepackage{bm}
\usepackage{empheq}
\usepackage{natbib}
\usepackage{cancel}
\usepackage{hyperref}
\usepackage[all]{hypcap}

\bibliographystyle{apj}

\newcommand{\Div}[1]{\ensuremath{\nabla\cdot\left( #1\right)}}
\newcommand{\angles}[1]{\ensuremath{\left\langle #1 \right\rangle}}
\newcommand{\grad}{\ensuremath{\nabla}}
\newcommand{\RB}{Rayleigh-B\'{e}nard }
\newcommand{\stressT}{\ensuremath{\bm{\bar{\bar{\Pi}}}}}


\begin{document}
%%%%% Create nice title and abstract
\author{Evan H. Anders}
\affil{University of Colorado -- Boulder}
\author{Benjamin P. Brown}
\affil{University of Colorado -- Boulder}
\title{Heat transport in stratified convection across mach number blahdy blah}
\begin{abstract}
EVAN DON'T FORGET TO DO THIS
\end{abstract}
\maketitle


%%%%% Body of the paper

\section{Motivation}
Blah Blah stuff.

\section{Model \& Equations}

\subsection{The Fully Compressible Navier-Stokes Equations}
We solve over the Fully Compressible Navier-Stokes equations with an energy-conserving energy equation,
which take the form:
\begin{align}
&\begin{aligned}
&\frac{\partial \rho}{\partial t} + \Div{\rho\bm{u}} = 0
	\label{continuity_eqn}
\end{aligned}\\
&\begin{aligned}
&\rho\left(\frac{\partial \bm{u}}{\partial t} + \left(\bm{u}\cdot\nabla\right)\bm{u}\right) =
-\grad P + \rho\bm{g} - \nabla\cdot\stressT
	\label{momentum_eqn}
\end{aligned}\\
&\begin{aligned}
&\rho c_V\left(\frac{\partial T}{\partial t} + \left(\bm{u}\cdot\grad\right)T + (\gamma-1)T\Div{\bm{u}}\right) \\
&\,\,\,\,\,\,\,\,\,\,\,\,\,\,\,\,\,\,
+ \Div{-\kappa\grad T} = -\left(\stressT\cdot\nabla\right)\cdot\bm{u} 
	\label{energy_eqn}
\end{aligned}
\end{align}
where the viscous stress tensor is defined such as
\begin{equation}
\Pi_{ij} \equiv -\mu\left(\frac{\partial u_i}{\partial x_j} + \frac{\partial u_j}{\partial x_i} - \frac{2}{3}\delta_{ij}\Div{\bm{u}}\right),
\end{equation}
where $\mu$ is the \emph{dynamic viscosity} (in units of [mass $\cdot$ length$^{-1}$ $\cdot$ time$^{-1}$]) and $\delta_{ij}$
is the kronecker delta function.  We also define the \emph{kinematic viscosity}, which has units of a classic diffusion coefficient,
as $\nu \equiv \mu/\rho$.

Taking the dot product of the velocity and the equation of momentum conservation (Eq. \ref{momentum_eqn}) and adding it to the energy
equation



\subsection{Control Parameters}

\bibliography{biblio.bib}
\end{document}
