%----------------------------------------------------------------------------------------
%	PACKAGES AND OTHER DOCUMENT CONFIGURATIONS
%----------------------------------------------------------------------------------------

\documentclass[aps, 11pt, singlecolumn]{revtex4-1} % Set the font size (10pt, 11pt and 12pt) and paper size (letterpaper, a4paper, etc)
\usepackage{natbib}
\bibliographystyle{apsrev}
\usepackage{setspace}

\begin{document}
%----------------------------------------------------------------------------------------
%	LETTER CONTENT
%----------------------------------------------------------------------------------------
\noindent
Dear Editor,
$\,$\newline

\newenvironment{myquotation}{
\begin{quotation}
\itshape
}{ 
\end{quotation}
}

\begin{singlespace}
We thank the two referees for their careful reading of our letter
and their detailed and constructive critique.
In responding to these two reports, the science within
our paper has been significantly strengthened.
Addressing the questions raised by report A led us to undertake
our first 3D runs, and we report on the results of these in the
revised draft.  In responding to report B,
we realized that our previous definition of the
Nusselt number (Nu) was flawed, and have since defined
a slightly modified, more appropriate Nu.  This new definition better unifies
the results at low and high Mach number (Ma) and makes it easier to understand
the changes between the low and high Ma regimes. We have streamlined our discussion
of the control parameters and have shown how evolved properties (e.g., Ma) depend
on the control parameters of our experiment.  Our previous presentation of the
experiment was confusing, and following the report's comments we feel that we
have improved the presentation.

We apologize for the three month delay between receiving the referee reports
and the return of the revised draft.  In these months we have
recomputed all solutions, designed and tested our 3D implementation, and
thought carefully about the proper definition of Nu.  Through this
process we have gained a greater understanding of the fundamental
fluid properties in our simulations.  


We include the revised version of our paper, ``Convective heat transport in stratified
atmospheres at low and high Mach number,'' in which we have addressed the critique contained
in both reports.  We think the science has been substantially improved, as has
the presentation of the experiment and the results.  The text and figures have been significantly
revised.  
Below we include a detailed response to the reports; we include the text of
the reports in their entirety along with our responses inline.  We address major
points of the critiques first and then address minor points.


$\,$\newline
\noindent
Sincerely,

Evan H. Anders \& Benjamin P. Brown



%\end{singlespace}

$\,$
\newline
$\,$
\newline
\noindent
\Large{\textbf{Response to report A:}}\newline$\,$\newline\indent
\large{\textbf{Major Points}}

\begin{myquotation}
The refereed paper is a reasonable step forward in an extremely
difficult and very important for understanding of fundamental physics
in planetary and star atmospheres problem of heat transfer in
convection of compressible, stratified media at moderately high Mach
numbers.

Authors expended previous studies to wider region of parameters, which
control an efficiency of the convection, and reported observation of
three regimes of convection with different scaling of the convection
efficiency (Nusselt number) vs. temperature gradient (Rayleigh number)
with a complicated time-space behavior of velocity, density and
temperature fluctuations.

Authors attacked this problem using high-end computer facilities of
NASA and efficient well justified numerical schemes. Nevertheless, to
be able to perform the direct numerical simulations of fully
compressible Navier-Stokes equations coupled with an equation for the
temperature transport, they restricted themselves by considering
two-dimensional version of the problem. Unfortunately, they did not
clarified, in which respect the two-dimensional simulations reflect
the real features of the three-dimensional physics. There is a well
known example (incompressible hydrodynamic turbulence, governed by the
Euler equation), in which an additional integral of motion in
two-dimensional case completely changes the basic physics of the
problem, including the direction of the energy flux over scales.
\end{myquotation}

We now have run and analyzed select 3D cases at both low and high $\epsilon$
at select values of the Rayleigh number.  These simulations have shown
us that many fluid properties (e.g., the density profile, Mach number, and Reynolds number)
of the evolved 3D state are similar to the 2D simulations.  We now report on these
limited 3D results as well as our more extensive 2D results. These results are shown
in Figs. 1, 3, and 4, with large symbols, and are discussed in the text.
Broadly, there seems to be some agreement between the 2D and 3D dynamics.

\begin{myquotation}
The Letter is clearly written with a well balanced general
introduction to the field, a formulation of particular simplifications
of basic equations of motion and a presentation of the results of
numerical simulations. According to my understanding, the paper should
be interesting for non-experts in the field. I tend to recommend this
manuscript for publication in PRL, after detailed clarification of the
relations between two- and three-dimensional description on the basics
of Rayleigh-Bernard convection and on their generalization to the
compressible case.
\end{myquotation}
\noindent
This concludes our response to report A.  We thank referee A for this report.




$\,$
\newline
$\,$
\newline
\noindent
\Large{\textbf{Response to report B:}}\newline$\,$\newline\indent
\large{\textbf{Major Points}}

\begin{myquotation}
This paper essentially examines the heat transport in compressible
convection as a function of Rayleigh number and Mach number for fixed
density stratification and Prandtl number via 2D simulations. The major
results cited are the scalings of the heat transport (measured by a
Nusselt number) with the Rayleigh number.

This paper has one substantial result and hints at another with
limited investigation. The major interesting result is that as flows
become supersonic, then the heat transport characteristics change
dramatically due to shock heating transported into the downflows. The
other aspect that is hinted at but not really elucidated, is that
compressible convection at high and low superadiabaticity is
substantially different.

The result regarding the heat transport characteristics and scaling
are sufficiently difficult to obtain and sufficiently interesting to
warrant publication. However, this paper is presented in a very
mysterious way that significantly confuses these interesting results.
The description of the modelling used is perplexing and misleading,
and the major results are buried. I will try to explain my issues
below. Overall, my feeling is that this paper needs substantial
re-writing to be publishable.
\end{myquotation}
We appreciate the honesty in report B, and we have come to agree that the
structure of our previous draft was confusing.  We have restructured the text
to clarify the presentation of both the experiment and the results. 
We have highlighted the major results of flow properties and how those
scale in low- and high- Mach number convection.  As can be seen in
the rest of our response and the revised letter, 
supersonic flows behave substantially similarly to other flows
under our new analysis.  Surprisingly,
compressible convection at high and low superadiabaticity is more similar
than different, in contrast to our initial analysis.

\begin{myquotation}
My first and major issue is that this paper presents the model as
though the Mach number is a parameter. The Mach number is a diagnostic
of compressible convection. Compressible convection is governed by 4
parameters: a measure of the density stratification, a measure of the
superadiabaticity, and two measures of the diffusivities (viscous and
thermal). These can all be considered as timescales. Since there is a
further timescale (the sound crossing time) available, these 4
parameters can be non-dimensionalised. Some of these
non-dimensionalisations can be cast in terms of a Rayleigh number and
a Prandtl number for convenience of comparison with Boussinesq models,
for example. The Mach number, on the other hand, is a derived,
diagnostic quantity that depends on the choice of these 4 parameters.
This is easily seen from the results in the paper, e.g. Figure 1. Here
the Mach number is a measured quantity plotted as a function of (a)
the Rayleigh number, and then (b) the superadiabaticity. The paper
here is therefore confusing because it casts the results in terms of
`low or high Mach number', as if this were a parameter. There
are many examples of this throughout the paper, but see the abstract,
and, for instance, in the introduction ` the two control parameters
of RB convection are joined by the degree of stratification, $n_\rho$,
across the domain and the characteristic Ma of the convective
flows'. What the authors generally mean in their writeup is that
they are either choosing a small or large superadiabaticity (epsilon)
at fixed Ra, or small or large Ra at fixed epsilon. The former is used
for most of the results section.  ...
\end{myquotation}
We agree with the report. We now present the experiment with $\epsilon$
as the control parameter, and we show that Ma is a strong function of
$\epsilon$ and a weaker function of Ra (Fig. 1).  In our studies, we
find and show that $\epsilon$ and Ra largely determine Ma and Re in the
evolved state, respectively (Fig. 1 \& 3).

\begin{myquotation}
...However, it is true that the
interesting change in transport results appear when the flow becomes
supersonic, but this could mean high superadiabaticity (epsilon) *or*
high Ra (for fixed Pr). The results should all really be cast as
`at high enough Ra or high enough superadiabaticity, a high Ma flow
results and this changes the transport characteristics'.
Ultimately, the scaling results exhibited are in terms of Ra, and this
makes perfect sense, although these results are only confined to a
small paragraph at the end of the results section without much
elaboration or explanation. I personally would like a much more causal
relationship explained between the shock heating and `spinners'
and the heat transport results.
\end{myquotation}
In revisiting our results, we found that the substantial differences between
low and high $\epsilon$ in heat transport arose from a poor definition of
Nu and from changes to the overall stratification.  We have determined a
more consistent version of Nu (based on evolved properties rather than initial
properties).  Now, high and low $\epsilon$ transport is substantially similar.
We do see a small change in transport when 2D simulations achieve a mean
Ma of O(1), and to test this we have conducted an additional suite of
experiments at higher $\epsilon = 1$.  In these solutions, all of the flow
properties show a change of behavior at the point at which the flows enter the
sonic regime.  This is shown in Figs. 1, 3, and 4 and discussed in the text.

Figure 4 shows a surprising result.  We have removed the former Fig. 3 (fluxes
as a function of height) so that we can discuss the result in our current
Fig. 4 fully.  We find that the density contrast across the domain at high $\epsilon$
appreciably shrinks in the evolved state compared to the initial state.  Furthermore, we find
that density inversions form near the boundary layers of these atmospheres once the mean atmospheric Mach number approaches
one, and show that this density inversion becomes significant at high Ra and high $\epsilon$,
and persists in 3D.  This is discussed fully in the Results \&
Discussion section.  We have combined the Results \& Discussion sections in the revised draft.


$\,$\newline\noindent
We have addressed the major points in the critique; we now continue with the
minor points and clarifications.

$\,$
\newline
$\,$
\newline
\indent
\large{\textbf{Minor Points}}\newline
\begin{myquotation}
Beyond this main issue, there are a lot of small things that I don't
understand, which I will try and list here, in chronological order.

`Numerical constraints ... to moderately high Ma': What is the
numerical constraint of low Ma flow?
\end{myquotation}
When the convection is low Ma, the convective dynamics are much slower
than the linear acoustic waves.  In explicit timestepping schemes, these
fast acoustic waves set the Courant–Friedrichs–Lewy (CFL) limit.
By employing modern IMEX schemes, we are able to implicitly evolve these
fast linear waves and timestep on the timescale of the slow convective
dynamics.    This is detailed in the second to
last paragraph of the Experiment section.

\begin{myquotation}
`RB' By `Rayleigh-Benard problem' here I assume that the
authors are referring to Boussinesq dynamics? It might be a good idea
to make this clear.
\end{myquotation}
We have clarified that we are referring to incompressible Boussinesq convection
in the Introduction.

\begin{myquotation}
`the Ma is controlled by the superadiabatic excess': It is clear
from Fig 1 that this is not a complete statement.
\end{myquotation}
We have clarified the discussion of how Ma scales with $\epsilon$ and
Ra throughout the text.

\begin{myquotation}
Experiment

Eqn (1): Better define co-ordinates, especially $z$ and $z_0$
\end{myquotation}
We agree that $z_0$ was confusing notation and this has been removed.  
We define $z$ shortly below Eqn (1).

\begin{myquotation}
The non-dimensionalization is very confusingly written. Can you write
out the non-dimensionalised polytope? This section also says that the
`timescale is the isothermal sound crossing time of the layer'
and then two sentences later says that `we use buoyancy time
units', so which is it? The definition of the latter does look like
the sqrt(epsilon) so maybe these are the same?
\end{myquotation}
We have clarified that the equations are non-dimensionalized on
the isothermal sound crossing timescale.  We have also clarified
how the evolution time of the flow is linked to the buoyancy time.
This appears in the final paragraph of the Experiment section.

\begin{myquotation}
`The scaling of the entropy gradient with epsilon ... evolved
values ': I really have no idea what these two sentences mean,
sorry!
\end{myquotation}
We agree that this sentence was confusing and have moved the
discussion of the magnitude of evolved fluctuations to the
discussion of Fig. 1.

\begin{myquotation}
Eqns: I don't understand why these equations are formulated with
gradients of nu when mu is constant. The stress tensor only depends on
mu and so this can be pulled out of all derivatives. Or am I missing
something? The case of constant nu and therefore variable mu is much
harder. Similarly for the formulation in terms of chi and not the
thermal conductivity; variable thermal conductivity is hard but
variable chi is easy. Why not write in terms of mu and k not nu and
chi?
\end{myquotation}
Our equations are formulated as the evolution of velocities rather
than momentum; this formluation is crucial to our current IMEX
timestepping of acoustic waves at low Mach numbers.  
In this velocity-based formulation, $\nu$ and $\chi$ are the
natural variables.  If $\nu$ and $\chi$ are constant in space,
the problem becomes a constant coefficient problem and is numerically
simple.  However, a spatially constant $\nu$ and $\chi$ does not satisfy
thermal equilibrium within the initial polytrope, which is only satisfied when
$\kappa_0 = \rho_0\chi$ is constant (where as in the letter, subscript 0
denotes initial values).  To satisfy thermal equilibrium requires
a gradient in $\chi$, and to satisfy a constant Pr through the domain we
must also have a gradient in $\nu$.  $\nu$ and $\chi$ are now nonconstant
coefficients, and this approach, as we take here, is much more challenging numerically. 
 We have
clarified and detailed the profiles of $\nu, \,\chi, \,\mu,\text{ and }\kappa$
in the second paragraph of the Experiment section.

\begin{myquotation}
`IMEX': I am assuming this acronym stands for implicit-explicit?
Maybe write out?
\end{myquotation}
We have spelled this out in the second to last paragraph of the Experiment section.

\begin{myquotation}
`extended to include variable nu and chi' : see above. This seems
very unnecessary!
\end{myquotation}
See above description of $\nu$ and $\chi$ in our equations.

\begin{myquotation}
Results

$d_z T_{ad} = 0$: This is only true for liquids not gases. See Speigel and
Veronis 1960.
\end{myquotation}
This is a good point that we had not appreciated.  Most of the studies of Boussinesq
convection with which we are familiar make the assumption that the fluid is incompressible.
It is true that in an ideal gas which is compressible but barely stratified, the
adiabatic temperature gradient is what we found and what is reported by
Spiegel and Veronis, $\partial_z T_{ad} = -g/c_P$ \cite{spiegel&veronis1960}.
We have updated the text following Eqn 8 to reflect this.

\begin{myquotation}
`While it has been suggested that pressure forces...': I do not
understand the discussion here. Do the authors regard the breakup of
the down flows as an extreme version of asymmetry?
\end{myquotation}
On further reflection, we realized that we are uncertain about how to
quantify asymmetries as the flows break up.  
Being unable to determine how $\epsilon$ affects symmetries in the
flow, we have removed this discussion.

\begin{myquotation}
`exhibit states in which the flux entering...': Surely in a
stationary state the flux in has to equal the flux out. Are the
authors saying that this is not a stationary state?
\end{myquotation}
As detailed in the revised text, the combination of fixed-temperature
boundary conditions and no-slip boundary conditions allows the system to alternate between
roll states of convection and shear states of convection.  These states persist over long
time scales (hundreds of buoyancy times) and the atmosphere is in flux disequilibrium in both states.
A sensible equilibrium and average emerges over long times.  This discussion appears with
the discussion of Fig. 3a.

\begin{myquotation}
Discussion

Can the authors demonstrate the basic principles of whatever balance
is described by the Grossman-Lohse theory?
\end{myquotation}
We believe that the Grossman-Lohse theory, which models the energy dissipation rate of both
kinetic and potential energies, could be extended to stratified, compressible flows which also
have potential energy.  At low $\epsilon$, it seems the extension would be easier to
develop. However, we are inexperienced in boundary layer theory and presently 
cannot say with certainty that an expansion of this theory to stratified convection is tractable. 
We have removed this suggestion from letter.  


\begin{myquotation}
`under solar conditions ... we expect that epsilon  $\approx 10^{-20}$
...`: Is this small epsilon a reflection of efficient mixing of
convection in the ultimate nonlinear state or of the initial linear
convective driving?
\end{myquotation}
This is a good question which we do not know the answer to.  As a result, we have
removed this large extrapolation of our results in Fig. 1 from the discussion
entirely.

$\,$\newline\noindent
This concludes our response to report B.  We thank referee B for this report.

$\,$
\newline
\noindent


\end{singlespace}




\bibliography{../../biblio.bib}
\end{document}
