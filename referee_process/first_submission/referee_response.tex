%%%%%%%%%%%%%%%%%%%%%%%%%%%%%%%%%%%%%%%%%
% Professional Formal Letter
% LaTeX Template
% Version 2.0 (12/2/17)
%
% This template originates from:
% http://www.LaTeXTemplates.com
%
% Authors:
% Brian Moses
% Vel (vel@LaTeXTemplates.com)
%
% License:
% CC BY-NC-SA 3.0 (http://creativecommons.org/licenses/by-nc-sa/3.0/)
%
%%%%%%%%%%%%%%%%%%%%%%%%%%%%%%%%%%%%%%%%%

%----------------------------------------------------------------------------------------
%	PACKAGES AND OTHER DOCUMENT CONFIGURATIONS
%----------------------------------------------------------------------------------------

\documentclass[aps, 11pt, singlecolumn]{revtex4-1} % Set the font size (10pt, 11pt and 12pt) and paper size (letterpaper, a4paper, etc)
\usepackage{natbib}
\bibliographystyle{apsrev}
\usepackage{setspace}

\begin{document}
%----------------------------------------------------------------------------------------
%	LETTER CONTENT
%----------------------------------------------------------------------------------------
\noindent
Dear Editor,
$\,$\newline

\begin{singlespace}
We have returned the revised version of our paper, 
``Convective heat transport in stratified atmospheres at low and high
Mach number,'' in which we have addressed the concerns of the referees.
We hope this version of the document is acceptable for publication.  We
have adjusted the text to address the criticism of the referee reports,
with a detailed description of our changes below.

We are grateful for the careful reading and responses of the referees.  
We thank referee A, whose
comments pushed us to undertake our first 3D runs in which we found the
transport results were nearly identical to the 2D results.
Upon reflecting on the
comments of referee B, we realized that the old definition of the
Nusselt number (Nu) that we were using was flawed, and have since defined
a slightly modified, more appropriate Nu.  This new definition better unifies
the results at low and high Mach number (Ma) and makes it easier to understand
the changes between the low and high Ma regimes.  Furthermore, 
the comments of referee B drove us to better analyze all properties of our evolved
atmospheres, and we have come to better understand how the Reynolds scales with
the Rayleigh number and the structure of the evolved atmosphere compared to the
initial atmosphere.

We apologize for the three month delay between receiving the referee reports
and the return of the revised draft.  In these months we have
recomputed all solutions, designed and tested our 3D implementation, and
thought carefully about the proper definition of Nu.  Through this
process we have gained a greater understanding of the fundamental
fluid properties in our simulations and how they change coherently
across parameter space.  We have removed the previous figure 3 from our
draft, as we feel it did not share any important information (the sum of
the fluxes adds up to a flat line in an equilibrated solution, and this
is well known).  The previous figure 4 has now become figure 3, and in addition
to showing the scaling of Nu with Ra, we now also show the scaling of the
Reynolds number, Re, with Ra.  We have added a new figure 4, in which we
find that the density contrast across the domain at high $\epsilon$ appreciably
shrinks in the evolved state compared to the initial state.  Furthermore, we
find that density inversions form near the boundary layers of these atmospheres
once the mean atmospheric Mach number approaches one, and show that this density
inversion becomes significant at high Ra and high $\epsilon$.
%\end{singlespace}

$\,$
\newline
$\,$
\newline

\textbf{Detailed response to referee report A:}

\begin{quotation}
The refereed paper is a reasonable step forward in an extremely
difficult and very important for understanding of fundamental physics
in planetary and star atmospheres problem of heat transfer in
convection of compressible, stratified media at moderately high Mach
numbers.

Authors expended previous studies to wider region of parameters, which
control an efficiency of the convection, and reported observation of
three regimes of convection with different scaling of the convection
efficiency (Nusselt number) vs. temperature gradient (Rayleigh number)
with a complicated time-space behavior of velocity, density and
temperature fluctuations.

Authors attacked this problem using high-end computer facilities of
NASA and efficient well justified numerical schemes. Nevertheless, to
be able to perform the direct numerical simulations of fully
compressible Navier-Stokes equations coupled with an equation for the
temperature transport, they restricted themselves by considering
two-dimensional version of the problem. Unfortunately, they did not
clarified, in which respect the two-dimensional simulations reflect
the real features of the three-dimensional physics. There is a well
known example (incompressible hydrodynamic turbulence, governed by the
Euler equation), in which an additional integral of motion in
two-dimensional case completely changes the basic physics of the
problem, including the direction of the energy flux over scales.
\end{quotation}

We have run and analyzed select 3D cases at both low and high $\epsilon$
at select values of the Rayleigh number.  These simulations have shown
us that the density profile, Mach number, and Reynolds number of the
evolved 3D state are nearly identical to the 2D simulations.  The Nusselt
number appears to change slightly between the 2D and 3D cases at low $\epsilon$, 
but the high $\epsilon$ results agree almost perfectly.  These results are shown
in Figs. 1, 3, and 4, and the agreement suggests that 2D cases
represent the bulk properties of the 3D dynamics well.

\begin{quotation}
The Letter is clearly written with a well balanced general
introduction to the field, a formulation of particular simplifications
of basic equations of motion and a presentation of the results of
numerical simulations. According to my understanding, the paper should
be interesting for non-experts in the field. I tend to recommend this
manuscript for publication in PRL, after detailed clarification of the
relations between two- and three-dimensional description on the basics
of Rayleigh-Bernard convection and on their generalization to the
compressible case.
\end{quotation}

Boussinesq theory and DNS results have shown that the scaling of Nu with Ra
is nearly the same in 2D and 3D for fluids with Pr $\geq$ 1 \cite{ahlers&all2009}.
We have shown that this is true in our results, and have mentioned this connection
to Boussinesq convection in the first paragraph of our discussion section.





$\,$
\newline
$\,$
\newline

\textbf{Detailed response to referee report A:}

\begin{quotation}
This paper essentially examines the heat transport in compressible
convection as as function of Rayleigh number and Mach number for fixed
density stratification and Prandtl number via 2D simulations The major
results cited are the scalings of the heat transport (measured by a
Nusselt number) with the Rayleigh number.

This paper has one substantial result and hints at another with
limited investigation. The major interesting result is that as flows
become supersonic, then the heat transport characteristics change
dramatically due to shock heating transported into the downflows. The
other aspect that is hinted at but not really elucidated, is that
compressible convection at high and low superadiabaticity is
substantially different.

The result regarding the heat transport characteristics and scaling
are sufficiently difficult to obtain and sufficiently interesting to
warrant publication. However, this paper is presented in a very
mysterious way that significantly confuses these interesting results.
The description of the modelling used is perplexing and misleading,
and the major results are buried. I will try to explain my issues
below. Overall, my feeling is that this paper needs substantial
re-writing to be publishable.
\end{quotation}
We thank referee B for their honesty regarding the confusing
structure of the paper.  We have tried to restructure the text in
a more sequential and hopefully less confusing way.

\begin{quotation}
My first and major issue is that this paper presents the model as
though the Mach number is a parameter. The Mach number is a diagnostic
of compressible convection. Compressible convection is governed by 4
parameters: a measure of the density stratification, a measure of the
superadiabaticity, and two measures of the diffusivities (viscous and
thermal). These can all be considered as timescales. Since there is a
further timescale (the sound crossing time) available, these 4
parameters can be non-dimensionalised. Some of these
non-dimensionalisations can be cast in terms of a Rayleigh number and
a Prandtl number for convenience of comparison with Boussinesq models,
for example. The Mach number, on the other hand, is a derived,
diagnostic quantity that depends on the choice of these 4 parameters.
This is easily seen from the results in the paper, e.g. Figure 1. Here
the Mach number is a measured quantity plotted as a function of (a)
the Rayleigh number, and then (b) the superadiabaticity. The paper
here is therefore confusing because it casts the results in terms of
`low or high Mach number', as if this were a parameter. There
are many examples of this throughout the paper, but see the abstract,
and, for instance, in the introduction ` the two control parameters
of RB convection are joined by the degree of stratification, $n_\rho$,
across the domain and the characteristic Ma of the convective
flows'. What the authors generally mean in their writeup is that
they are either choosing a small or large superadiabaticity (epsilon)
at fixed Ra, or small or large Ra at fixed epsilon. The former is used
for most of the results section.  ...
\end{quotation}
It is true that $\epsilon$ is the primary control parameter we are
referring to when we mentioned ``high'' or ``low'' Mach number. This
is because Ma is a very strong function of $\epsilon$, and a weaker
function of Ra, as shown in Fig. 1.  At the ranges of Ra which we are
reasonable to examine via DNS (up to nearly $10^7$Ra$_{\text{crit}}$),
the Ma increases by less than a factor of 100 from simulations at
onset to simulations near the state-of-the-art.  Thus, the choice of
very low $\epsilon$ ($10^{-4}$, $10^{-7}$) forces all simulations at
that parameter into the Ma $\ll 1$ regime, regardless of Ra.  Whereas the
choice of $\epsilon \approx 1$ forces all simulations into the Ma $\approx 1$
regime.

This being said, it is understandable that the manner in which we presented
Ma and $\epsilon$ was somewhat confusing.  We have changed the text to try
to more explicitly state that $\epsilon$ is the control parameter, and that
setting $\epsilon$ to be very small allows us to study low Ma convection,
whereas setting $\epsilon$ near 1 allows us to study high Ma convection.  We
have chosen to retain the title of the letter while adding clarification to
the text.

\begin{quotation}
...However, it is true that the
interesting change in transport results appear when the flow becomes
supersonic, but this could mean high superadiabaticity (epsilon) *or*
high Ra (for fixed Pr). The results should all really be cast as
`at high enough Ra or high enough superadiabaticity, a high Ma flow
results and this changes the transport characteristics'.
Ultimately, the scaling results exhibited are in terms of Ra, and this
makes perfect sense, although these results are only confined to a
small paragraph at the end of the results section without much
elaboration or explanation. I personally would like a much more causal
relationship explained between the shock heating and `spinners'
and the heat transport results.

Beyond this main issue, there are a lot of small things that I don't
understand, which I will try and list here, in chronological order.

Introduction

'Numerical constraints ... to moderately high Ma': What is the
numerical constraint of low Ma flow?
\end{quotation}
When the convection is low Ma, the convective dynamics are much slower
than the linear acoustic waves.  By stepping this linear component
implicitly, we bypass the high Ma component of these waves and timestep
on the order of the convective dynamics.

\begin{quotation}
'RB' By `Rayleigh-Benard problem' here I assume that the
authors are referring to Boussinesq dynamics? It might be a good idea
to make this clear.
\end{quotation}
We have reworded this sentence at the beginning of paragraph 3 to make it
clear that we are referring to Boussinesq convection.

\begin{quotation}
'the Ma is controlled by the superadiabatic excess': It is clear
from Fig 1 that this is not a complete statement.
\end{quotation}
We have removed this sentence from the introduction, and mentioned that
the superadiabaticity is the important control parameter.

\begin{quotation}
Experiment

Eqn (1): Better define co-ordinates, especially $z$ and $z_0$
\end{quotation}
$z_0$ was confusing and has been removed.  We define $z$ shortly
below the eqn (1).

\begin{quotation}
The non-dimensionalization is very confusingly written. Can you write
out the non-dimensionalised polytope? This section also says that the
`timescale is the isothermal sound crossing time of the layer'
and then two sentences later says that `we use buoyancy time
units', so which is it? The definition of the latter does look like
the sqrt(epsilon) so maybe these are the same?
\end{quotation}
Ben, what is he asking us to write out?
In our numerical simulations, one time unit is the isothermal sound
crossing time at the top of the layer.  However, convective dynamics
have a typical overturn time which can be a few simulation time units
(at high $\epsilon$) or it can have an overturn time of many simulation
time units (at low $\epsilon$).  The buoyancy time allows us to estimate this.

\begin{quotation}
'The scaling of the entropy gradient with epsilon ... evolved
values ': I really have no idea what these two sentences mean,
sorry!
\end{quotation}
The initial entropy gradient of the polytrope is $\nabla S \propto -\epsilon$,
and as a result evolved variables are $O(\epsilon)$.  We have removed this
sentence, as it is confusing, and said this in the results section.

\begin{quotation}
Eqns: I don't understand why these equations are formulated with
gradients of nu when mu is constant. The stress tensor only depends on
mu and so this can be pulled out of all derivatives. Or am I missing
something? The case of constant nu and therefore variable mu is much
harder. Similarly for the formulation in terms of chi and not the
thermal conductivity; variable thermal conductivity is hard but
variable chi is easy. Why not write in terms of mu and k not nu and
chi?
\end{quotation}
$\mu$ is initially constant, which means that $\mu = \nu\rho_0$
must be constant.  Since we are in a stratified atmosphere, this
means that $\nu \propto 1/\rho_0$.  A similar relation between
$\kappa$ and $\chi$ exists.  Thus the gradients of $\nu$ and $\chi$
are present.  $\mu$ and $\kappa$ are allowed to evolve with the
density profile, and we have discussed this more clearly in section X.

\begin{quotation}
'IMEX': I am assuming this acronym stands for implicit-explicit?
Maybe write out?
\end{quotation}
We have spelled this out in the paragraph after the paragraph
containing Eq. 7.

\begin{quotation}
'extended to include variable nu and chi' : see above. This seems
very unnecessary!
\end{quotation}
See above discussion

\begin{quotation}
Results

$d_z T_{ad} = 0$: This is only true for liquids not gases. See Speigel and
Veronis 1960.
\end{quotation}
This is a good point that we had not appreciated.  Most of the studies of Boussinesq
convection with which we are familiar make the assumption that the fluid is incompressible.
It is true that in an ideal gas which is compressible but barely stratified, the
adiabatic temperature gradient is what we found and what is reported by
Spiegel and Veronis, $\partial_z T_{ad} = -g/c_P$, \cite{spiegel&veronis1960}.
We have updated the text following eqn 8 to reflect this.

\begin{quotation}
'While it has been suggested that pressure forces...': I do not
understand the discussion here. Do the authors regard the breakup of
the down flows as an extreme version of asymmetry?
\end{quotation}
We are merely stating that at low Ma, pressure fluctuations are small and
tend to not be positive in the upper atmosphere and negative in the lower atmosphere
regardless of whether we are observing an upflow or a downflow.  The classical argument
\cite{hurlburt&all1984} of ``buoyancy breaking accelerates downflows and deccelerates
upflows'' seems to break down when the Mach number is small from the cases we've studied.

\begin{quotation}
'exhibit states in which the flux entering...': Surely in a
stationary state the flux in has to equal the flux out. Are the
authors saying that this is not a stationary state?
\end{quotation}
We are saying this is a non-stationary state.  The combination of fixed-temperature
boundary conditions and no-slip boundary conditions allows the system to fluctuate between
roll states of convection and shear states of convection.  These states persist over long
time scales (hundreds of buoyancy times) and the atmosphere is in flux disequilibrium in both states.
We have tried to update the text to make this clearer.

\begin{quotation}
Discussion

Can the authors demonstrate the basic principles of whatever balance
is described by the Grossman-Lohse theory?
\end{quotation}
The Grossman-Lohse theory for Boussinesq dynamics models the energy dissipation rate of both
kinetic and potential energies, and separates these rates within the boundary layers and
the bulk flow.  Stratified, compressible convection must also deal with potential energy,
and as the Mach number becomes large and density inversions begin to occur (such as in our new
Fig. 4), a more complex theory involving dissipation rates of all three energies could
in practice be developed.  In the low Ma case in which the flows are small and density 
gradients monitonically point downwards, it is likely that a simplified version of the
Grossman-Lohse theory could be developed in the context of polytropically stratified atmospheres.
Here, the density profile would mean that the upper and lower boundary layers become
asymmetrical.

Unfortunately, we are not experts in boundary layer theory and will need to educate ourselves
further on how the Grossman-Lohse theory was developed and applied before we can say anything
with certainty in stratified convection. In keeping with this, we have removed this nod from
the Discussion section and made suggestions towards further experiments which we are certain our
work can inform.


\begin{quotation}
'under solar conditions ... we expect that epsilon  $\approx 10^{-20}$
...': Is this small epsilon a reflection of efficient mixing of
convection in the ultimate nonlinear state or of the initial linear
convective driving?
\end{quotation}
These approximations in the initial letter were a representation of the choices of
parameters necessary to achieve the desired Mach number and Rayleigh number of convective
motions in the deep solar interior.  There, Ma $\approx 10^{-4}$ and Ra $\approx 10^{20}$,
and due to the scaling of Mach number with $\epsilon$ and Ra, we made a best-guess estimate
based on our scaling laws in fig. 1 for what $\epsilon$ could describe that low Ma at such
extreme values of Ra.

\end{singlespace}


\noindent
Sincerely,

Evan H. Anders \& Benjamin P. Brown



\bibliography{../../biblio.bib}
\end{document}
