%----------------------------------------------------------------------------------------
%	PACKAGES AND OTHER DOCUMENT CONFIGURATIONS
%----------------------------------------------------------------------------------------

\documentclass[aps, 11pt, singlecolumn]{revtex4-1} % Set the font size (10pt, 11pt and 12pt) and paper size (letterpaper, a4paper, etc)
\usepackage{natbib}
\bibliographystyle{apsrev}
\usepackage{setspace}

\begin{document}
%----------------------------------------------------------------------------------------
%	LETTER CONTENT
%----------------------------------------------------------------------------------------
\noindent
Dear Editor,
$\,$\newline

\newenvironment{myquotation}{
\begin{quotation}
\itshape
}{ 
\end{quotation}
}

\begin{singlespace}
We thank the two referees for their detailed and constructive critique.
We have completed further 3D runs which make the scaling of parameters at 3D 
more clear to the reader.
We have modified the text of the paper to address the critiques of
both reports, and have adjusted the styles of select figures in order to
more efficiently call out interesting results.

We include the revised version of our paper.  Below, we include a
detailed response to the reports; we include the text of the reports
in their entirety along with our responses inline.

$\,$\newline
\noindent
Sincerely,

Evan H. Anders \& Benjamin P. Brown



%\end{singlespace}

$\,$
\newline
$\,$
\newline
\noindent
\Large{\textbf{Response to report B:}}\newline$\,$\newline\indent

\begin{myquotation}

The second submission of this paper, after the first round of review,
is SUBSTANTIALLY different from the first time around, and essentially
needs refereeing from scratch as a new paper. The paper does still
essentially examine the heat transport in compressible convection as
as function of superadiabaticity and Rayleigh number (and therefore
varying thermal diffusivity) for fixed density stratification and
Prandtl number via 2D simulations The new paper adjusted the
formulation of the Nusselt number, changing the results substantially,
and added some 3D simulations to the previously all-2D simulations.
The major result is the scalings of the heat transport (measured by
their Nusselt number) with the Rayleigh number bears a remarkable
similarity to standard Boussinesq Rayleigh-Benard convection, at all
the wide range of superadiabaticities simulated. This in itself is an
interesting result, as it somewhat surprising considering that their
fixed background stratification has quite a strong density contrast
($\sim$20) and some of their departures from adiabaticity are large
($\epsilon \sim$ 1 $\rightarrow$ adiabatic index m=0.5).

I think this paper is getting closer to publishable, although I still
find the explanation of the setup of the model in particular somewhat
opaque though, and the interesting results not really addressed in any
detail, as I'll explain below.

I am glad that the authors took my comments on the setup from before
at least partially to heart! I still think things could be clearer.
Most specifically again, is the question of what are the salient
parameters of compressible convection. There are 4, as mentioned
previously, and the authors here have chosen to keep one fixed -- a
measure of the stratification, for which they use $n_\rho$. For the
others, it is nice to cast things in terms of a Rayleigh number and a
Prandtl number since this gels nicely with standard Boussinesq
Rayleigh-Benard convection (RBC). However, then the third parameter,
the superadiabaticity, which the authors call $\epsilon$, is also part of
the definition of the Rayleigh number, and therefore, as it should be,
the Rayleigh parameter is really a definition of the thermal
diffusivity. To the unaware reader more familiar with RBC, this can be
a bit confusing, since there are no independent measures of the
driving (superadiabaticity) and the thermal diffusivity there. It
would really help the reader here to point out the following in
relation to the simulation sets that are performed:

\begin{itemize}
\item At fixed $\epsilon$, varying Ra means that the thermal diffusivity $\chi_t$
scales like 1/$\sqrt{\text{Ra}}$.
\item At fixed Ra, varying $\epsilon$ means that the thermal diffusivity $\chi_t$
scales like $\epsilon$.
\item Since Pr=fixed, viscous diffusivity scales like thermal diffusivity.
\item (and of course, all vary with depth individually)
\end{itemize}
\end{myquotation}
We have reworked the discussion of defining Ra and Pr, and have added
a new Eqn (3) and related discussion which demonstrates explicitly 
how our timescales vary with input parameters.  
We have also explicitly stated the two types
of experiments we perform (varying $\epsilon$ while holding Ra constant,
and vice versa), and what these experiments do to the timescales of
our problem.

\begin{myquotation}
Beyond that, I think everything is right. The expression for the
polytopes in terms of the number of density scale heights makes the
notation a little over-complex. There are some disconnects in where
the non-dimensionalisation is performed (they say that ``take R=1'' at
one point and then non-dimensionalise R out again later, for example).
The description of which of the diffusivities or the
conductivity/dynamic viscosity are constant could certainly be
tightened up.
\end{myquotation}
We agree. We have moved the discussion of non-dimensionalization after the
definition of the polytrope and before the discussion of diffusivities.


\begin{myquotation}
With regard to the results:

In light of what I said above, the results in Fig 2. are not
surprising. Higher Ra here at fixed epsilon means lower thermal
diffusivity, and therefore eddies might be expected to retain their
identity longer. The above explanation would help.
\end{myquotation}
We agree and Eqn. 3 now shows this clearly.
We have revised our discussion of Fig. 2c such that it 
explicitly states that long-lasting eddies are what we expect based on 
input parameters.

\begin{myquotation}
There is quite a bit of mention of “windy” states without any real
technical description of what they are. Please either give more
information (at least something visual to distinguish from non-windy
states) or remove the distraction.
\end{myquotation}
We have removed the three scattered discussions of ``windy'' states,
and consolidated them to a short discussion in the paragraph describing
Fig. 1.  Figs. 1, 2, and 3 have been updated to make it clear which
runs exhibited ``windy'' states, and we think including them in the
discussion at this level may be useful to the field when comparing to
the work of Goluskin or others.

\begin{myquotation}
Nusselt number: Usually this is a ratio of heat transport in the
turbulent state to that in the conduction state. So therefore, aren’t
the two $F_A$’s different on the top and the bottom? The top has a
modified kappa but the bottom has the original kappa profile?
\end{myquotation}
The evolution of $\kappa$ with $\rho$ (see discussion between
Eqns 2 \& 3) means that the evolved atmosphere has a different
conductive state than the initial atmosphere.  $\rho$ evolves
the most at large $\epsilon$ (Fig. 4).
Thus, both the numerator and denominator of Eqn. (9) 
use the same evolved state and
value of $F_A$.  In the first version of the letter, 
$F_A$ was instead defined by the initial atmosphere, and
the $F_{\text{cond}}$ used in the denominator of Eqn. (9) was
based on the initial atmosphere as well, as traditionally
in Rayleigh-Benard.  This led to the
large deviation in Nu scaling at high and low $\epsilon$,
as a result of the large shifts in the density profile.
However, as Fig. 4 shows, this shift was merely a result of the
density profile changing rather than a difference in the
amount of flux being carried.

\begin{myquotation}
I think the paragraphs on the Nu vs Ra and the Re/Pe vs Ra are the
meat of the paper! It would really be nice here to know what causes
the difference between 2D and 3D in the Nu plots. The 2/7 law is often
associated with more “windy” states. That seems LESS likely in 3D, so
what is going on? Note also that the sensitive dependence on the exact
roll or other structure seems to imply that the simulation box is too
small. This dependence should not be the case.
\end{myquotation}
We also are unsure of what the 2/7 scaling relationship means for these
fully compressible states.  We are interested in exploring this and potential
boundary layer theories in future work, but a full examination of this
is beyond the scope of this letter.
We have conducted a very small number of high aspect ratio solutions which
do not appear to converge to vastly different flow morphology.  We mention this
in the discussion of Fig. 3a.


\begin{myquotation}
The Re vs Ra are a bit mysterious at low Ra. I would expect the Re $\sim$
Ra$^{1/2}$. Can you explain these alternative scalings or at least why the
expected scaling emerges at large Ra? Are the 2D ones just
over-constrained?
\end{myquotation}
The increased scaling of Re with Ra is now explained in the text at the end
of the discussion of Fig 3b.  2D solutions build over-constrained spinners
like flywheeling convection in Rayleigh-Benard, which don't seem to appear
in our 3D solutions at these parameters.

\begin{myquotation}
Regarding Fig 4 -- are the authors calculating the integrated evolved
density profile to get this number? If the effect is only in the
boundary layer as they mention, I am surprised that the deviation is
so large. This figure needs much more explanation!
\end{myquotation}
We have expanded our discussion of these two measurements when we discuss Fig. 4 in the text.
If density were monotonic, these measurements would be equivalent, but density
inversions form within the boundary layers and the max/min are found away from the boundaries.
We have observed inverted boundary layers at the top of
the domain which span more than one density scale height at $\epsilon = 1$.
These are interesting and it's surprising that they persist in 3D.

~\newline

\noindent
This concludes our response to report B.  We thank referee B for this second report.




$\,$
\newline
$\,$
\newline
\noindent
\Large{\textbf{Response to report C:}}\newline$\,$\newline\indent

\begin{myquotation}
This Letter reports results from a numerical study of thermal
convection in a compressible fluid at unity Prandtl number and a range
of initial stratification. The main result is the scaling of the
Nusselt number with the Rayleigh number, supplemented by some
observations such as characteristic changes in the transition between
subsonic and supersonic regimes.

This problem is highly challenging. Ideally, a numerical exploration
should be guided by some theoretical insights or practical
observations. The present paper appears to be ``purely'' numerical
without such guidance, albeit some comparison with the well-known
problem of incompressible Rayleigh--Bernard convection. Nevertheless,
the results are significant and should be published. While I do not
wholeheartedly recommend this paper to PRL, I would not object to its
acceptance.

Below is a minor comment that the authors may find useful.

Most parts of the paper is relatively well written, except in the
introduction. Here, there is room for improvement. The introductory
paragraph is virtually empty. I would cite a couple of papers for
"These prior studies" and be specific about both "important insight"
and known "fundamental properties". The final paragraph of the
introduction has failed to fulfill its mission. I would explicitly
state (at least) the most significant result of the paper in this
paragraph.
\end{myquotation}
We have clarified the structure of the introduction paragraph.  We included
citations to prior work at high Ma and set the context for the present study
of fundamental heat transport at both low and high Ma.
The
most important result of the paper is called out in the last paragraph of the
introduction.

$\,$\newline\noindent
This concludes our response to report C.  We thank referee C for this report.

$\,$
\newline
\noindent


\end{singlespace}




\bibliography{../../biblio.bib}
\end{document}
