%\documentclass[iop]{emulateapj}
\documentclass[aps, prl, twocolumn, groupedaddress]{revtex4-1}
\usepackage{amsmath}
\usepackage{graphicx}
\usepackage{epstopdf}
\usepackage{listings}
\usepackage{color}
\usepackage{bm}
\usepackage{empheq}
\usepackage{natbib}
\usepackage{cancel}
\usepackage[colorlinks=True, linkcolor=blue, citecolor=blue]{hyperref}
\usepackage[all]{hypcap}

\bibliographystyle{apsrev}

\newcommand{\Div}[1]{\ensuremath{\nabla\cdot\left( #1\right)}}
\newcommand{\angles}[1]{\ensuremath{\left\langle #1 \right\rangle}}
\newcommand{\grad}{\ensuremath{\nabla}}
\newcommand{\RB}{Rayleigh-B\'{e}nard }
\newcommand{\stressT}{\ensuremath{\bm{\bar{\bar{\Pi}}}}}


\begin{document}
%%%%% Create nice title and abstract
\author{Evan H. Anders}
\affiliation{Department of Astrophysical \& Planetary Sciences, University of Colorado -- Boulder}
\affiliation{Laboratory for Atmospheric and Space Physics, Boulder, CO}
\author{Benjamin P. Brown}
\affiliation{Department of Astrophysical \& Planetary Sciences, University of Colorado -- Boulder}
\affiliation{Laboratory for Atmospheric and Space Physics, Boulder, CO}
\title{Heat transport in stratified convection across mach number blahdy blah}

\begin{abstract}
This is where an abstract defining what we're doing with stratified convection and Nusselt numbers will go.
\end{abstract}
\maketitle


%%%%% Body of the paper
\section{Introduction \label{section:intro}}
Compressible convection is ubiquitous among natural systems such as stellar envelopes and planetary atmospheres.
The widely-studied \RB convection in which up- and downflows are symmetric and the Rayleigh number (Ra, the
ratio of buoyant driving to diffusive damping) serves as the primary control parameter lacks a number of the
key features which make studies of stratified convection difficult to master.  Early studies of stratified
convection in two \cite{graham1975, chan&all1982, hurlburt&all1984, cattaneo&all1990} and three 
\cite{malagoli&all1990, cattaneo&all1991, brummell&all1996} dimensions
utilized polytropically stratified atmospheres, in which the temperature gradient, thermal diffusivity, and
gravity are constant throughout the depth of the atmosphere and the atmosphere is in hydrostatic equilibrium.
While perhaps not perfectly realistic and often abandoned for atmospheres constructed using more realistic
radiative transfer properties (cite some papers), the polytrope is a particularly useful reference state for
stratified convection studies.

A canonical value of the ``polytropic index'' used in the study of convection of monatomic ideal gases is
$m = 1$, such that the density and temperature profile are both linear.  Recently, \cite{brandenburg&all2005}
argued that as the polytropic index approaches the adiabatic value of $m_{ad} = 1.5$, the convective flux of
the system becomes negligible compared to the background flux gradient, and the Nusselt number of the
system approaches zero.  They argue that, as a result, values of $m \rightarrow -1$ are ideal polytropes
to study, as the radiative flux approaches zero and the convective flux carries the whole of the system
flux in equilibrium. 

Here, we argue for The definition of a new Nusselt number, where the radiative flux is defined relative to
the \emph{adiabatic} state rather than to a linear temperature profile.  We demonstrate that values
of $m$ close to adiabatic show a Nusselt number (similar?) to \cite{johnston&doering2009} and that
such values of $m$ are useful in allowing us to probe low-Mach number convection, such as that
deep within the Sun's convection zone (cite a paper for that?)


\section{Model \& Equations \label{section:model}}
We study a fluid whose equation of state is that of an ideal gas, $P = R^*\rho T$ and whose
initial stratification is polytropic, where
\begin{equation}
\begin{split}
\rho(z) &= \rho_{00}(z_0 - z)^m \\
T(z)    &= T_{00}(z_0 - z)^m
\end{split}
\end{equation}
and $z$ increases upwards. We define the height of the atmosphere such that $n_\rho$ density scale heights fit in the
atmosphere, or such that $\ln[\rho(L_{z})/\rho(0)] = n_\rho$.  We nondimensionalize our equations such that 
$P = \rho = T = 1$ at the top of the atmosphere, requiring $z_0 \equiv L_z + 1$ and $R^* = T_{00} = \rho_{00} = 1$.  
We specify the entropy gradient at the top of the atmosphere as $\grad S(L_z) = -\epsilon$ such that the
polytropic index of the atmosphere is $m = (\gamma - 1)^{-1} - \epsilon = m_{ad} - \epsilon$.

We evolve the Fully Compressible Navier-Stokes equations with an energy-conserving energy equation,
which take the form:
\begin{align}
&\begin{aligned}
&\frac{D \ln\rho}{D t} + \Div{\bm{u}} = 0
	\label{eqn:continuity_eqn}
\end{aligned}\\
&\begin{aligned}
&\rho\frac{D\bm{u}}{D t}=
-\grad P + \rho\bm{g} - \nabla\cdot\stressT
	\label{eqn:momentum_eqn}
\end{aligned}\\
&\begin{aligned}
\rho c_V\left(\frac{D T}{D t} + (\gamma-1)T\Div{\bm{u}}\right) + &\Div{-\kappa\grad T} = \\
&-\left(\stressT\cdot\nabla\right)\cdot\bm{u} 
	\label{eqn:energy_eqn}
\end{aligned}
\end{align}
where $D/Dt \equiv \partial_t + \bm{u}\cdot\grad$ and the viscous stress tensor is defined as
\begin{equation}
\Pi_{ij} \equiv -\mu\left(\frac{\partial u_i}{\partial x_j} + \frac{\partial u_j}{\partial x_i} - \frac{2}{3}\delta_{ij}\Div{\bm{u}}\right),
	\label{eqn:stress_tensor}
\end{equation}
where $\mu$ is the \emph{dynamic viscosity} (in units of [mass $\cdot$ length$^{-1}$ $\cdot$ time$^{-1}$]) and $\delta_{ij}$
is the kronecker delta function.  We also define the \emph{kinematic viscosity}, which has units of a classic diffusion coefficient,
as $\nu \equiv \mu/\rho$.  In a similar fashion, we define the thermal diffusivity, $\chi \equiv \kappa / \rho$ [note: Kundu has a $c_P$ on
the bottom, here].

Dotting Eq. \ref{eqn:momentum_eqn} with $\bm{u}$ and adding it to Eq. \ref{eqn:energy_eqn}, we retrieve the full energy equation in
conservation form,
\begin{equation}
\begin{split}
\frac{\partial}{\partial t}&\left(\rho\left[\frac{|\bm{u}|^2}{2} + c_V T + \phi\right]\right) +\\
&\Div{\rho\bm{u}\left[\frac{|\bm{u}|^2}{2} + h + \phi\right] + \bm{u}\cdot\stressT - \rho\chi\grad T} = 0,
	\label{eqn:energy_eqn_full}
\end{split}
\end{equation}
where $h \equiv c_V T + P/\rho$ is the system enthalpy and $\phi = -gz$ is the gravitational potential.  All of the 

We use initial conditions of randomized temperature perturbations on the order of $10^{-6}$ below the background
temperature field (EVAN -- TRY TO DO 1E-6*EPSILON). We impose stress free, impenetrable, constant temperature boundary
conditions.

The efficiency of convection is defined by the Nusselt number.  While the Nusselt number is well-defined in \RB convection
as the amount of total convective and radiative flux divided by the steady-state background flux \cite{johnston&doering2009, otero&all2002},
a well-defined Nusselt number is more elusive in stratified convection.  We suggest that a more general definition of the
Nusselt number is similar to that defined by \RB convection, but where both the numerator and denominator have the
radiative flux of the adiabatic temperature gradient of the atmosphere removed.  Such a definition was first used
by \cite{hurlburt&all1984} who wrote it as
\begin{equation}
N \equiv \frac{F_T - F_A}{F - F_A},
\end{equation}
where $F_T$ is the sum of fluxes in the system, $F_A = \kappa g / c_P$ is the radiative flux of the corresponding adiabatic profile
of the atmosphere, and $F = \Delta T / Lz$ is the radiative flux carried by a linear temperature profile across the domain.  Under
the specific case of constant temperature boundary conditions, $\Delta T$ is constant and the denominator of the Nusselt number
reduces to $F - F_A = -\kappa\epsilon \grad T_0 (\gamma-1)/\gamma$.  Thus, the denominator of the Nusselt number is directly proportional
to the level of superadiabaticity of the background atmosphere, as is the time-dependent value of $F_T - F_A$.  Under the Boussinesq
approximation of \RB convection, the lack of density variation means that the temperature field is directly linked to entropy, and
the adiabatic temperature gradient is definitionally zero.  In the case of \cite{brandenburg&all2005}, the case of
$\epsilon \rightarrow (m_{ad} + 1)$ is examined, but this also corresponds to the case where $g \rightarrow 0$ and the
corresponding adiabatic atmospheric temperature gradient is isothermal.

\subsection{Control Parameters}
In addition to the explicitly scaling the level of superadiabaticity of the background entropy gradient via $\epsilon$,
we utilize the nondimensional Prandtl (Pr) and Rayleigh (Ra) numbers to determine our atmospheric diffusivities.  Pr
is the ratio of kinematic viscosity to thermal diffusivity, which we set equal to one.  Ra is the ratiof buoyant driving
to diffusivity of the form
\begin{equation}
\text{Ra} \equiv \frac{g L_z^3 (\Delta S/c_P)}{\nu\chi},
	\label{eqn:ra_def}
\end{equation}
where $\Delta S/c_P$ is the entropy difference between the top and bottom boundaries of the reference state.  Note that
$\nu \propto \chi \propto \rho^{-1}$ such that Ra increases with depth as the square of the density.  In this work we
specify Ra at the top of the domain.

\begin{acknowledgements}
This work was supported by the CU/NSO Hale Graduate Fellowship blah.
\end{acknowledgements}

\bibliography{../biblio.bib}
\end{document}
