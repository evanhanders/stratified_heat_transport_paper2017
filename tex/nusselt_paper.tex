%\documentclass[iop]{emulateapj}
\documentclass[aps, prl, twocolumn, groupedaddress]{revtex4-1}
\usepackage{amsmath}
\usepackage{graphicx}
\usepackage{epstopdf}
\usepackage{listings}
\usepackage{color}
\usepackage{bm}
\usepackage{empheq}
\usepackage{natbib}
\usepackage{cancel}
\usepackage[colorlinks=True, linkcolor=blue, citecolor=blue]{hyperref}
\usepackage[all]{hypcap}

\bibliographystyle{apsrev}

\newcommand{\Div}[1]{\ensuremath{\nabla\cdot\left( #1\right)}}
\newcommand{\angles}[1]{\ensuremath{\left\langle #1 \right\rangle}}
\newcommand{\grad}{\ensuremath{\nabla}}
\newcommand{\RB}{Rayleigh-B\'{e}nard }
\newcommand{\stressT}{\ensuremath{\bm{\bar{\bar{\Pi}}}}}


\begin{document}
%%%%% Create nice title and abstract
\author{Evan H. Anders}
\affiliation{Department of Astrophysical \& Planetary Sciences, University of Colorado -- Boulder}
\affiliation{Laboratory for Atmospheric and Space Physics, Boulder, CO}
\author{Benjamin P. Brown}
\affiliation{Department of Astrophysical \& Planetary Sciences, University of Colorado -- Boulder}
\affiliation{Laboratory for Atmospheric and Space Physics, Boulder, CO}
\title{Heat transport in stratified convection across mach number blahdy blah}

\begin{abstract}
This is where an abstract defining what we're doing with stratified convection and Nusselt numbers will go.
\end{abstract}
\maketitle


%%%%% Body of the paper
\section{Introduction \label{section:intro}}
Talk about \cite{hurlburt&all1984, brummell&all1996, cattaneo&all1991},  and then talk about some
\RB stuff, such as \cite{johnston&doering2009}.

\section{Model \& Equations \label{section:model}}
We study a fluid whose equation of state is that of an ideal gas, $P = R^*\rho T$ and whose
initial stratification is polytropic, where
\begin{equation}
\begin{split}
\rho(z) &= \rho_{00}(z_0 - z)^m \\
T(z)    &= T_{00}(z_0 - z)^m
\end{split}
\end{equation}
and $z$ increases upwards. We define the height of the atmosphere such that $n_\rho$ density scale heights fit in the
atmosphere, or such that $\ln[\rho(L_{z})/\rho(0)] = n_\rho$.  We nondimensionalize our equations such that 
$P = \rho = T = 1$ at the top of the atmosphere, requiring $z_0 \equiv L_z + 1$ and $R^* = T_{00} = \rho_{00} = 1$.  
We specify the entropy gradient at the top of the atmosphere as $\grad S(L_z) = -\epsilon$ such that the
polytropic index of the atmosphere is $m = (\gamma - 1)^{-1} - \epsilon = m_{ad} - \epsilon$.

We evolve the Fully Compressible Navier-Stokes equations with an energy-conserving energy equation,
which take the form:
\begin{align}
&\begin{aligned}
&\frac{D \ln\rho}{D t} + \Div{\bm{u}} = 0
	\label{eqn:continuity_eqn}
\end{aligned}\\
&\begin{aligned}
&\rho\frac{D\bm{u}}{D t}=
-\grad P + \rho\bm{g} - \nabla\cdot\stressT
	\label{eqn:momentum_eqn}
\end{aligned}\\
&\begin{aligned}
\rho c_V\left(\frac{D T}{D t} + (\gamma-1)T\Div{\bm{u}}\right) + &\Div{-\kappa\grad T} = \\
&-\left(\stressT\cdot\nabla\right)\cdot\bm{u} 
	\label{eqn:energy_eqn}
\end{aligned}
\end{align}
where $D/Dt \equiv \partial_t + \bm{u}\cdot\grad$ and the viscous stress tensor is defined as
\begin{equation}
\Pi_{ij} \equiv -\mu\left(\frac{\partial u_i}{\partial x_j} + \frac{\partial u_j}{\partial x_i} - \frac{2}{3}\delta_{ij}\Div{\bm{u}}\right),
	\label{eqn:stress_tensor}
\end{equation}
where $\mu$ is the \emph{dynamic viscosity} (in units of [mass $\cdot$ length$^{-1}$ $\cdot$ time$^{-1}$]) and $\delta_{ij}$
is the kronecker delta function.  We also define the \emph{kinematic viscosity}, which has units of a classic diffusion coefficient,
as $\nu \equiv \mu/\rho$.  In a similar fashion, we define the thermal diffusivity, $\chi \equiv \kappa / \rho$ [note: Kundu has a $c_P$ on
the bottom, here].

Dotting Eq. \ref{eqn:momentum_eqn} with $\bm{u}$ and adding it to Eq. \ref{eqn:energy_eqn}, we retrieve the full energy equation in
conservation form,
\begin{equation}
\begin{split}
\frac{\partial}{\partial t}&\left(\rho\left[\frac{|\bm{u}|^2}{2} + c_V T + \phi\right]\right) +\\
&\Div{\rho\bm{u}\left[\frac{|\bm{u}|^2}{2} + h + \phi\right] + \bm{u}\cdot\stressT - \rho\chi\grad T} = 0,
	\label{eqn:energy_eqn_full}
\end{split}
\end{equation}
where $h \equiv c_V T + P/\rho$ is the system enthalpy and $\phi = -gz$ is the gravitational potential.  All of the 

We use initial conditions of randomized temperature perturbations on the order of $10^{-6}$ below the background
temperature field (EVAN -- TRY TO DO 1E-6*EPSILON). We impose stress free, impenetrable, constant temperature boundary
conditions.

\subsection{Control Parameters}
In addition to the explicitly scaling the level of superadiabaticity of the background entropy gradient via $\epsilon$,
we utilize the nondimensional Prandtl (Pr) and Rayleigh (Ra) numbers to determine our atmospheric diffusivities.  Pr
is the ratio of kinematic viscosity to thermal diffusivity, which we set equal to one.  Ra is the ratiof buoyant driving
to diffusivity of the form
\begin{equation}
\text{Ra} \equiv \frac{g L_z^3 (\Delta S/c_P)}{\nu\chi},
	\label{eqn:ra_def}
\end{equation}
where $\Delta S/c_P$ is the entropy difference between the top and bottom boundaries of the reference state.  Note that
$\nu \propto \chi \propto \rho^{-1}$ such that Ra increases with depth as the square of the density.  In this work we
specify Ra at the top of the domain.

\begin{acknowledgements}
This work was supported by the CU/NSO Hale Graduate Fellowship blah.
\end{acknowledgements}

\bibliography{../biblio.bib}
\end{document}
