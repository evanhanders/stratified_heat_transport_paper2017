%\documentclass[iop]{emulateapj}
\documentclass[aps, prl, twocolumn, nofootinbib, groupedaddress, amsfonts, amssymb, amsmath]{revtex4-1}
\usepackage{graphicx}
\usepackage{bm}
\usepackage{natbib}
%\usepackage[colorlinks=True, linkcolor=blue, citecolor=blue]{hyperref}
%\usepackage[all]{hypcap}

\bibliographystyle{apsrev}

\newcommand{\Div}[1]{\ensuremath{\nabla\cdot\left( #1\right)}}
\newcommand{\angles}[1]{\ensuremath{\left\langle #1 \right\rangle}}
\newcommand{\grad}{\ensuremath{\nabla}}
\newcommand{\RB}{Rayleigh-B\'{e}nard }
\newcommand{\stressT}{\ensuremath{\bm{\bar{\bar{\Pi}}}}}
\newcommand{\lilstressT}{\ensuremath{\bm{\bar{\bar{\sigma}}}}}
\newcommand{\nrho}{\ensuremath{n_{\rho}}}
\newcommand{\approptoinn}[2]{\mathrel{\vcenter{
	\offinterlineskip\halign{\hfil$##$\cr
	#1\propto\cr\noalign{\kern2pt}#1\sim\cr\noalign{\kern-2pt}}}}}

\newcommand{\appropto}{\mathpalette\approptoinn\relax}

\newcommand\mnras{{MNRAS}}%
          % Monthly Notices of the RAS

\begin{document}
%%%%% Create nice title and abstract
\author{Evan H. Anders}
\affiliation{Department of Astrophysical \& Planetary Sciences, University of Colorado -- Boulder}
\affiliation{Laboratory for Atmospheric and Space Physics, Boulder, CO}
\author{Benjamin P. Brown}
\affiliation{Department of Astrophysical \& Planetary Sciences, University of Colorado -- Boulder}
\affiliation{Laboratory for Atmospheric and Space Physics, Boulder, CO}
\title{Convective heat transport in stratified atmospheres at low and high Mach number}

\begin{abstract}
Convection in astrophysical systems is stratified and
often occurs at high Rayleigh number (Ra) and low
Mach number (Ma).
Here we study stratified convection in the context of 
plane-parallel, polytropically stratified atmospheres. 
We hold the density stratification ($n_{\rho}$), aspect
ratio, and Prandtl 
number (Pr) constant while varying the superadiabaticity
and Ra.  We find that Ma is a strong function of the superadiabaticity
in both 2D and 3D simulations, and that it is a 
weak function of Ra in 2D.
In 2D simulations, the scaling of the Nusselt Number (Nu) and the Reynolds number (Re)
with Ra weaken
once the mean Ma of the solution is unity, but in 3D
Re and Nu seem to have consistent scaling across Ra.
We further find that when the superadiabaticity and Ma are large,
density inversions form in the thermal boundary layers.
%As Ra increases and $\text{Ma} \rightarrow 1$, a scaling 
%of Nu $\propto$ Ra$^{0.45}$ is observed.  
%As Ra increases to a regime where Ma $\geq 1$,
%this scaling gives way to a weaker Nu $\propto$ Ra$^{0.19}$. 
%In the regime of Ma $\ll 1$, a consistent
%Nu $\propto$ Ra$^{0.31}$ is retrieved,  reminiscent of the 
%Nu $\propto$ Ra$^{2/7}$ seen in \RB convection.
\end{abstract}
\maketitle


%%%%% Body of the paper
\section{Introduction}
\refstepcounter{section}
\label{sec:intro}
Convection transports energy in stellar and planetary atmospheres.
In these objects, flows are compressible and
feel the atmospheric stratification.  While in some systems this stratification 
is negligible, it is significant in regions such as
the convective envelope of the Sun, which spans 14 density scale heights.
In the bulk of these systems, flows are at very low Mach Number (Ma), 
but numerical constraints have restricted most studies of 
compressible convection to high Ma.
These studies have provided important insight into the low temperature, 
high Ma region near the Sun's surface, but few fundamental
properties of low Ma convection which characterize deeper motions
are known.

%Convective heat transport is essential
%in the cores of high mass stars,
%the envelopes of low mass stars, and the atmospheres of terrestrial and 
%jovian planets.  These astrophysical systems have density stratifications
%ranging from a few density scale heights (e.g. massive star cores)
%up to 14 density scale heights in the convective envelopes of low
%mass stars such as the Sun.
%Atmospheric stratification presumably modifies the fundamental
%dynamics and heat transport mechanisms of these atmospheres.
%Understanding the fundamental
%properties of compressible convection in stratified media is essential
%to characterizing systems in astrophysics and planetary sciences.  
%Numerical constraints have often restricted studies of compressible
%convection to moderately high Mach number (Ma), appropriate to regions near 
%the Sun's surface.  Few fundamental properties of
%low Ma convection, which occurs in the deep solar interior, are known.

In the widely-studied \RB (RB) problem of incompressible Boussinesq convection, 
upflows and downflows are symmetrical, and
the conductive flux ($\propto \grad T$) approaches 
zero in the convective interior.
Early numerical experiments of moderate-to-high Ma compressible convection
in two \cite{graham1975, chan&all1982,
hurlburt&all1984, cattaneo&all1990} and three 
\cite{cattaneo&all1991, brummell&all1996} dimensions
revealed that these two hallmark characteristics of RB convection change
significantly when stratification is included.  Downflow lanes
become fast and narrow, and upflow lanes turn into broad, slow upwellings.
Furthermore, the \emph{entropy} gradient is negated by convection 
rather than the temperature gradient, and
a significant conductive flux can exist in the presence of
efficient convection.

In RB convection, there exist two primary dynamical control parameters: 
the Rayleigh number (Ra, the ratio of
buoyant driving to diffusive damping) and the Prandtl number 
(Pr, the ratio of viscous to thermal
diffusivity). These numbers control two useful
measures of turbulence in the evolved solution:
the Reynolds
number (Re, the strength of advection to viscous diffusion)
and the Peclet number (Pe = Re Pr).  Stratified atmospheres
with negative entropy gradients are unstable to convection.
The magnitude of the unstable entropy gradient, or the superadiabaticity,
joins Ra and Pr as an important control parameter.  This 
\emph{superadiabatic excess} \cite{graham1975}, $\epsilon$, which 
sets the scale of the atmospheric entropy gradient,
primarily controls the Ma of the evolved solution.

%In RB convection, there exist two primary control parameters: 
%the Rayleigh number (Ra, the ratio of
%buoyant driving to diffusive damping) and the Prandtl number 
%(Pr, the ratio of viscous to thermal
%diffusivity).  These numbers coupled with the aspect ratio of 
%the physical domain and the boundary conditions
%determine the dynamics of the convection.  In stratified atmospheres, 
%in addition to specifying the equation of state and
%fundamental properties of the gas, the two control parameters of 
%RB convection are joined by the degree of
%stratification across the domain and the characteristic 
%Ma of the convective flows.  
%Polytropically stratified atmospheres, such as those used in 
%early studies \cite{graham1975, chan&all1982, hurlburt&all1984, 
%cattaneo&all1990, cattaneo&all1991, brummell&all1996}, are an ideal extension of
%RB convection into the stratified realm because the two additional 
%control parameters are directly linked to
%basic properties of the atmosphere.  The density stratification is 
%set by the number of density scale heights (\nrho)
%the atmosphere spans, and Ma is controlled 
%by the superadiabatic excess ($\epsilon$),
%the deviation of the polytropic index from the adiabatic polytropic 
%index \cite{graham1975}.

In this letter we study the behavior of convective heat transport, 
quantified by the Nusselt number (Nu), in plane-parallel, 
two- and three-dimensional, polytropically stratified atmospheres.  
We vary $\epsilon$ and Ra while holding Pr, aspect ratio, boundary conditions,
and initial atmospheric stratification
constant.  We also examie the behavior of flow speeds, as quantifed by Ma,
and the level of turbulence in solutions, as quantified by Re.
%We describe experimental methods in section 
%\ref{sec:experiment}, including the construction of atmospheres, 
%equations, and numerical methods.  
%Results are described in section \ref{sec:results} 
%and their implications are discussed
%in section \ref{sec:discussion}.

\section{Experiment} 
\refstepcounter{section}
\label{sec:experiment}
We examine a monatomic ideal gas with an adiabatic index of
$\gamma = 5/3$ whose equation of state is $P = R\rho T$.
This is consistent with the approach used in earlier work
and is the simplest stratified extension of RB.
We study atmospheres which are initially polytropically stratified,
\begin{equation}
\begin{split}
\rho_0(z) &= \rho_{t}(1 + L_z - z)^m, \\
T_0(z)    &= T_{t}(1 + L_z - z),
\label{eqn:polytrope}
\end{split}
\end{equation}
where $m$ is the polytropic index and $L_z$ is the depth of the atmosphere.
The height coordinate, $z$, increases upwards in the range $[0, L_z]$.  We
specify the depth of the atmosphere, $L_z = e^{n_{\rho}/m} - 1$, by choosing
the number of density scale heights, $n_{\rho}$, it spans.
Throughout this letter we set $n_{\rho} = 3$.  Satisfying hydrostatic
equilibrium sets the value of gravity, $g = R T_t (m + 1)$, which is
constant with depth.  We study atmospheres with aspect
ratios of 4 where both the $x$ and $y$ coordinates have the range $[0, 4L_z]$.
In our 2D cases, we only consider $x$ and $z$.

%
%We examine the simplest stratified extension of RB by studying a
%fluid composed of monatomic ideal gas particles with an adiabatic 
%index of $\gamma = 5/3$ and whose equation of state is $P = R\rho T$. 
%This is consistent with the approach used in earlier work.
%The initial atmosphere is a plane-parallel polytrope in which 
%the gravitational acceleration and conductive flux, 
%$\bm{F}_{\text{cond,0}} = -\kappa \partial_z T_0$, do not vary with depth. To
%achieve the latter condition, both $\kappa$ and $\partial_z T_0$ are constant.
%Under these assumptions, satisfying hydrostatic 
%equilibrium produces a stratification of
%\begin{equation}
%\begin{split}
%\rho_0(z) &= \rho_{t}(z_0 - z)^m, \\
%T_0(z)    &= T_{t}(z_0 - z),
%\label{eqn:polytrope}
%\end{split}
%\end{equation}
%where $m = m_{\text{ad}} - \epsilon$ is the polytropic index.
%The adiabatic polytropic index is $m_{\text{ad}} \equiv (\gamma-1)^{-1}$, and
%the superadiabatic excess is $\epsilon$ which sets the scale 
%of the entropy gradient ($\partial_z S_0 \propto -\epsilon$).
%A significant advance of this work is the ability to study large 
%and small values of $\epsilon$, as will be discussed.
%Thermodynamic variables are nondimensionalized at the 
%top of the atmosphere as  $P_0(L_z) = \rho_0(L_z) = T_0(L_z) = 1$, 
%requiring $z_0 \equiv L_z + 1$ and $R = T_{t} = \rho_{t} = 1$.
%By this choice, the non-dimensional length scale is the inverse 
%temperature gradient scale and the timescale is the isothermal 
%sound crossing time of this unit length.
%The height $z$ increases upwards within $[0, L_{z}]$, 
%where $L_{z} = e^{n_{\rho}/m} - 1$ is
%determined by $n_\rho$ and $\epsilon$.
%The characteristic timescale of convective dynamics
%is related to the atmospheric buoyancy time, 
%$t_{\text{b}} = \sqrt{L_z/g\epsilon}$, with $g = (m+1)$.
%Throughout this letter, we use buoyancy time units and 
%choose $n_{\rho} = 3$ such that the 
%initial density varies by a factor of 20.
%All atmospheres studied here have an aspect ratio of 4, 
%such that $L_x = 4L_z$.
%

Convective dynamics are controlled by the superadiabaticity
of the atmospheres as well as the atmospheric diffusivities.
The superadiabaticity, or the magnitude of the (negative) 
entropy gradient, is set by 
the superadiabatic excess, $\epsilon = m_{ad} - m$, where 
$m_{ad} = (\gamma - 1)^{-1}$.
The atmospheric thermal diffusivity, $\chi$,
and kinematic viscosity (or viscous diffusivity), $\nu$,
are determined by the Rayleigh number (Ra) and the Prandtl
number (Pr).  We set $\text{Pr } = \nu/\chi = 1$ throughout
this work. The polytropic initial conditions are in
thermal equilibrium, $\kappa_0 \partial_z T_0 =$ const,
so $\kappa_0 = \chi \rho_0$ must be constant. To keep
Pr constant with height, we set $\chi = \chi_t/\rho_0$
and $\nu = \nu_t/\rho_0$.  Under these constraints, the
diffusivity profiles are specified by choosing Ra
at the top of the domain,
\begin{equation}
\text{Ra}_{\text{top}} = \frac{g L_z^3 (\Delta S_0 / c_P)}{\nu_t\chi_t},
\end{equation}
where $\Delta S_0 = \epsilon\ln (L_z + 1)$ is the entropy difference 
between the top and bottom boundaries and
$c_P = R\gamma(\gamma-1)^{-1}$ is the specific heat 
at constant pressure. The profiles of $\nu$ and $\chi$ are constant with
time, and Ra at the bottom of the atmosphere is greater than Ra$_{\text{top}}$
by a factor of $e^{2n_{\rho}}$.  This formulation leaves
the thermal conductivity, $\kappa = \rho\chi$, and
the dynamic viscosity, $\mu = \rho\chi$, free to evolve
as the density profile changes.  Throughout this letter,
we will use Ra and Ra$_{\text{top}}$ interchangeably.

We decompose our thermodynamic variables such that $T = T_0 + T_1$ and
$\ln\rho = (\ln\rho)_0 + (\ln\rho)_1$, 
and the velocity $\bm{u} = u\hat{x} + v\hat{y} +  w\hat{z}$ 
has no background component.  In our 2D runs, $v = 0$.
We evolve the Fully Compressible Navier-Stokes equations,
\begin{align}
&\begin{aligned}
&\frac{\partial \ln\rho}{\partial t} + \grad\cdot\bm{u} 
    = -\bm{u}\cdot\grad\ln\rho,
	\label{eqn:continuity_eqn}
\end{aligned}\\
&\begin{aligned}
\frac{\partial\bm{u}}{\partial t} + \grad T - 
&\nu\grad\cdot\lilstressT - \lilstressT\cdot\grad\nu = \\
&-\bm{u}\cdot\grad\bm{u} - T\grad\ln\rho + \bm{g} + 
\nu\lilstressT\cdot\grad\ln\rho,
\label{eqn:momentum_eqn}
\end{aligned}\\
&\begin{aligned}
\frac{\partial T}{\partial t} -\frac{1}{c_V}\left(\right.\chi&\left.
    \grad^2 T + \grad T\cdot\grad\chi\right) = \\
	&-\bm{u}\cdot\grad T - (\gamma-1)T\grad\cdot{\bm{u}} \\
	&+ \frac{1}{c_V}\left(\chi\grad T \cdot\grad\ln\rho +
	\nu\left[\lilstressT\cdot\nabla\right]\cdot\bm{u}\right), 
	\label{eqn:energy_eqn}
\end{aligned}
\end{align}
with the viscous stress tensor given by
\begin{equation}
\sigma_{ij} \equiv \left(\frac{\partial u_i}{\partial x_j} + 
\frac{\partial u_j}{\partial x_i} - \frac{2}{3}\delta_{ij}\grad\cdot\bm{u}\right).
	\label{eqn:stress_tensor}
\end{equation}
Taking an inner product of
(\ref{eqn:momentum_eqn}) with $\bm{u}$ and adding it to 
(\ref{eqn:energy_eqn}) reveals the full energy equation,
\begin{equation}
\frac{\partial}{\partial t}\left(\rho\left[\frac{|\bm{u}|^2}{2} + c_V T + \phi\right]\right) +
\Div{\bm{F}_{\text{conv}} + \bm{F}_{\text{cond}}} = 0,
	\label{eqn:energy_eqn_full}
\end{equation}
where
$
\bm{F}_{\text{conv}} \equiv \bm{F}_{\text{enth}} + \bm{F}_{\text{KE}} + \bm{F}_{\text{PE}} + \bm{F}_{\text{visc}}
$
is the convective flux and $\bm{F}_{\text{cond}} = -\kappa \grad T$
is the conductive flux.
The individual contributions to $\bm{F}_{\text{conv}}$ are the enthalpy flux, 
$\bm{F}_{\text{enth}} \equiv \rho\bm{u}(c_V T + P/\rho)$;
the kinetic energy flux, 
$\bm{F}_{\text{KE}} \equiv \rho|\bm{u}|^2\bm{u}/2$;
the potential energy flux,
$\bm{F}_{\text{PE}} \equiv \rho\bm{u}\phi$ (with $\phi \equiv -gz$);
and the viscous flux, 
$\bm{F}_{\text{visc}} \equiv -\rho\nu\bm{u}\cdot\lilstressT$, and each 
must be considered. 
Understanding how these fluxes interact  
is crucial in characterizing convective heat transport.

We utilize the 
Dedalus\footnote{http://dedalus-project.org/} \cite{burns&all2016} 
pseudospectral framework to time-evolve  
(\ref{eqn:continuity_eqn})-(\ref{eqn:energy_eqn}) 
using an implicit-explicit (IMEX), third-order, four-step 
Runge-Kutta timestepping scheme RK443 \cite{ascher&all1997}.  
Variables are time-evolved on a dealiased Chebyshev (vertical)
and Fourier (horizontal, periodic) domain in which the
physical grid dimensions are 3/2 the size of the coefficient grid.  
Grid sizes range from
64x256 coefficient points at the lowest values of 
Ra to 1024x4096 coefficient points at Ra $> 10^{7}$ in 2D,
and from 64x128$^2$ to  256x512$^2$ in 3D. 
By using IMEX timestepping, we implicitly step the 
stiff linear acoustic wave contribution and are able to
efficiently study flows at moderate ($\approx 1$) 
and very low ($\approx 10^{-4}$) Ma.  Our equations take the form
of the FC equations in \cite{lecoanet&all2014}, extended to include
$\nu$ and $\chi$ which vary with depth, and we follow the approach there.
This IMEX approach has been successfully 
tested against a nonlinear benchmark  of the compressible 
Kelvin-Helmholtz instability \cite{Lecoanet_et_al_2016_KH}.

We impose impenetrable, stress free, fixed temperature boundary conditions at
the top and bottom of the domain such that 
$w = \partial_z u = T_1 = 0$ at $z = \{0, L_z\}$. 
$T_1$ is initially filled with
random white noise whose magnitude is infinitesimal
compared to $T_0$ and $\epsilon$.
We filter this noise spectrum in coefficient space, 
such that 25\% of the coefficients
have power. We non-dimensionalize our computational domains by setting
all thermodynamic variables to unity at $z = L_z$ by choosing
$R = T_t = \rho_t = 1$.  By this choice, the non-dimensional
grid length scale is the inverse temperature gradient scale and the 
simulation timescale is the isothermal sound crossing time, 
$\tau_I$, of this unit length.
Meaningful convective dynamics occur on 
timescales of the atmospheric buoyancy time,
$t_b = \tau_I \sqrt{L_z/g\epsilon}$. 
All reported results are taken from time averages
over many buoyancy times beginning 100$t_b$ 
after the start of our simulations in order to
assure our results are not biased by the convective transient.

\section{Results}
\refstepcounter{section}
\label{sec:results}


\begin{figure}[t]
\includegraphics[width=3.4375in]{./figs/ma_v_Ra.png}
\caption{The mean adiabatic Mach number of long-time-averaged profiles
is shown.  Error bars show the full range of Ma over the depth of the
atmosphere.
(a) Ma is plotted as a function of $\epsilon$, the superadiabatic excess,
at various values of Ra.
(b) Ma is plotted as a function of Ra at various values of $\epsilon$.
Larger symbols with white dots designate 3D runs.
\label{fig:ma_v_eps} }
\end{figure}


Solutions were time-evolved until a long time average of the fluxes
showed little
variance with depth. By performing a linear stability analysis, 
we determined that the onset of convection in 2D
occurs at $\text{Ra}_{\text{crit}} = \{11.15, 10.06, 10.97, 10.97\}$ 
for $\epsilon = \{1.0, 0.5, 10^{-4}, 10^{-7}\}$, respectively.  These
onset values hold in 3D.
We studied Rayleigh
numbers $\lesssim \left\{10^7, 10^5, 10^4, 10^3\right\}$Ra$_{\text{crit}}$ 
for $\epsilon = \{0.5, 10^{-4}, 1.0, 10^{-7}\}$, respectively. 

We find that the Mach number is a strong function of 
$\epsilon$ and a weak function of Ra.  
In our 2D runs, when Ma $< 1$ we observe a scaling law of
Ma(Ra$, \epsilon) \propto \epsilon^{1/2}$Ra$^{1/4}$, 
but this relation breaks down as the mean
Ma approaches 1 and the flows become supersonic (see Fig. \ref{fig:ma_v_eps}).  
In 3D, Ma appears to be a function of $\epsilon$ alone, with
Ma $\propto \epsilon^{1/2}$.
Furthermore, the value of $\epsilon$ sets the size of the
evolved thermodynamic variables, such that
$T_1/T_0 \propto \epsilon$ and $\rho_1/\rho_0 \propto \epsilon$.  
These fluctuations are very small when $\epsilon$ and Ma are small,
but can be O(1) for values of $\epsilon$ near $m_{\text{ad}}$.

\begin{figure}[t]
\includegraphics[width=3.4375in]{./figs/snapshots_fig.png}
\caption{Characteristic entropy fluctuations in evolved 2D flows roughly
140$t_b$ after the start of simulations. 
The time- and horizontally-averaged profile is removed in all cases.  
(a) An $\epsilon = 10^{-4}$, low Ma flow. (b) An $\epsilon = 0.5$, flow,
in which shock systems can be seen in the upper atmosphere (e.g.
$(x, z) \approx (30, 18-19), (70, 15-19))$.
(c) A high Ra, $\epsilon = 0.5$ case.  Convective structures become
much smaller scale, and shock systems can be see near $(x, z) \approx (30, 16-19),
(65, 17-19)$.
\label{fig:entropy_snapshots} }
\end{figure}

Low Ma flows (e.g. $\epsilon = 10^{-4}$)
display the classic narrow downflow and broad upflow lanes of stratified
convection (Fig. \ref{fig:entropy_snapshots}a).
While it has been suggested that buoyancy breaking as 
a result of pressure forces leads to
the asymmetric nature of up- and downflows
\cite{hurlburt&all1984}, at low $\epsilon$ this 
effect seems to be unimportant, and the convective structures form
as a result of flows obeying mass conservation while traversing
the stratified medium.  
Our choice of fixed-temperature, stress free boundary conditions 
allows the flux at the boundaries to vary, while also allowing
for domain-wide horizontal velocity profiles along the boundaries.
2D runs at $\text{Ra }> 10^3$Ra$_{\text{crit}}$
and $\epsilon = 10^{-4}$ exhibit long-term states of
flux disequilibrium, in which roll solutions such as those pictured in 
Fig. \ref{fig:entropy_snapshots}a
are punctuated by shearing states similar to those previously
reported in two-dimensional RB convection \cite{goluskin&all2014}.  
During both of these states, the system is in flux disequilibrium.
A long time average
over both of these states produces a flat flux profile which
can be sensibly analyzed, but the presence of these time-dependent states
makes it difficult to measure fluid quantities in these higher-Ra
cases at low $\epsilon$.

At high Ma (e.g. $\epsilon = 0.5$), bulk thermodynamic structures are similar but
shock systems form in the upper atmosphere near downflow lanes 
for sufficiently large values of $\epsilon$ and Ra (Fig. \ref{fig:entropy_snapshots}b,c).
Similar shock phenomena were reported in
both two \cite{cattaneo&all1990} and 
three \cite{malagoli&all1990} dimensional polytropic simulations previously.
As Ra is increased to large values 
(Fig. \ref{fig:entropy_snapshots}c), thermodynamic structures 
no longer span the whole domain but rather break up into 
small eddies which traverse the domain multiple
times before diffusing.  

\begin{figure}[t]
\includegraphics[width=3.4375in]{./figs/re_and_nu_v_Ra.png}
\caption{
(a) Variation of Nu as Ra increases at high and low $\epsilon$. 
Errors bars indicate the properly normalized range of the
value of Nu as a function of depth and
large error bars indicate a poorly converged solution.
(b) Variation of Re as Ra increases at high and low $\epsilon$.
Re is measured at the midplane of the atmosphere.
Larger symbols with white dots designate 3D runs.
 \label{fig:re_and_nu_v_ra}.
}
\end{figure}

The efficiency of convection is quantified by the Nusselt number (Nu).  
Nu is well-defined in RB convection
as the total flux normalized by the steady-state conductive flux 
\cite{johnston&doering2009, otero&all2002}.
In stratified convection Nu is more difficult to define, and we use
a modified version of a traditional stratified Nusselt number 
\cite{graham1975,hurlburt&all1984},
\begin{equation}
\text{Nu} \equiv \frac{\langle F_{\text{conv,z}} + F_{\text{cond,z}} - F_{\text{A}}\rangle}
{\langle F_{\text{cond,z}} - F_{\text{A}}\rangle} 
= 1 + \frac{\langle F_{\text{conv,z}}\rangle}{\langle F_{\text{cond,z}} - F_{\text{A}} \rangle}
\label{eqn:nusselt}
\end{equation}
where $F_{\text{conv,z}}$ and $F_{\text{cond,z}}$ are the 
z-components of $\bm{F}_{\text{conv}}$ and $\bm{F}_{\text{cond}}$,
and $\langle \rangle$ are volume averages.  
$F_{\text{A}} \equiv -\langle\kappa\rangle \partial_z T_{\text{ad}}$ 
is the conductive flux of the proper corresponding adiabatic atmosphere in
in thermal equilibrium.
For a compressible ideal gas in hydrostatic equilibrium,
$\partial_z T_{\text{ad}} \equiv - g / c_{P}$ \cite{spiegel&veronis1960}.  
It is important to measure the evolved value of
$\langle \kappa \rangle = \langle \rho\chi \rangle$, which is nearly
$\kappa_0$ when $\epsilon$ is small but can change appreciably for large
values of $\epsilon$.
In incompressible Boussinesq convection, where $\grad S = 0$ only when 
$\grad T = 0$, this definition reduces to the traditional definition
of the Nusselt number \cite{otero&all2002, johnston&doering2009}.

The variation of Nu with Ra is shown in 
Fig. \ref{fig:re_and_nu_v_ra}a.
In 2D and at low to moderate Ra, 
Nu $\propto$ Ra$^{1/3}$ regardless of $\epsilon$,
reminiscent of scaling laws in classical \RB theory \cite{king&all2012}.
At large Ra, Nu $\propto$ Ra$^{1/5}$ 
at $\epsilon = 0.5$.
The scaling of Nu with Ra 
is unclear at low $\epsilon$ once Ra is sufficiently large
for shearing states to occur.  It is also important to note that
the value of Nu is heavily dependent upon the specific thermodynamic
structures of the solution, as double roll states will transport
heat more efficiently than single roll states, and slight changes in
Ra can result in a solution latching on to one solution or the other.
In 3D, it appears that Nu $\propto$ Ra$^{2/7}$, a classic scaling law
seen in RB studies \cite{johnston&doering2009}.

The Reynolds number (Re = $|\bm{u}|L_z/\nu$) and Peclet number
(Pe = Pr Re)
compare the importance of advection to diffusion in the evolved
convective state.  For Pr = 1, such as in this work, Pe = Re.  
Our choice of $\{\nu,\chi\}\propto \rho_0^{-1}$ drastically changes
the value of Re between the top and bottom of the atmosphere.  We report values of
Re at the midplane ($z=L_z/2$) of the atmosphere in
Fig. \ref{fig:re_and_nu_v_ra}b.
In 2D and at low Ra, Re $\propto$ Ra$^{3/4}$.  At high Ra and $\epsilon$, 
where the average Ma $\approx 1$, 
this scaling gives way to a Re $\propto$ Ra$^{1/2}$,
where all changes in Re are due to the lowering of the diffusivity
through raising Ra.
For large values of Ra but at low $\epsilon$, there appears to be a scaling
of Re $\propto$ Ra$^{2/3}$, but once again the shearing states have made
these points difficult to measure.  In 3D, the scaling of Re appears to
consistently be Re $\propto$ Ra$^{1/2}$, consistent with the high-$\epsilon$,
high Ra results seen in 2D and the lack of scaling of Ma with Ra in Fig. 
\ref{fig:ma_v_eps}b.

\begin{figure}[t]
\includegraphics[width=3.4375in]{./figs/density_v_ra.png}
\caption{\label{fig:nrho_v_ra} The stratification of 
evolved solutions is measured
in two ways.  Solid symbols show $\ln(\rho(z=0)/\rho(z=L_z))$, 
the density contrast
as measured at the upper and lower boundary.  The empty symbols show 
$\ln(\text{max}(\rho)/\text{min}(\rho))$. 
Unsurprisingly, at low $\epsilon$ the evolved
$n_{\rho}$ is not different from the initial conditions to first order.  
At high $\epsilon$,
the density contrast shrinks, and once the mean 
Ma approaches 1 (as in Fig. \ref{fig:ma_v_eps}b), the two methods of
measuring the density stratification bifurcate as density 
inversions form within the thermal
boundary layers. Larger symbols with white dots represent 3D runs.}
\end{figure}

As the thermodynamic variables converge to their steady state values, 
the density profile evolves while remaining in hydrostatic equilibrium 
to zeroth order.  In Fig. \ref{fig:nrho_v_ra} we show the number of density 
scale heights present in the evolved solution using two measures.  
We find that in 2D, once the average Ma of the domain
becomes approximately one, large density inversions begin to 
form in the boundary layers, as was reported by \cite{brandenburg&all2005}.  
The agreement of Nu across $\epsilon$ (Fig. \ref{fig:re_and_nu_v_ra}a), 
particularly at low Ra in which all four of our cases collapse onto a single
power law, is striking in light of the vastly different evolved stratifications
felt by the flows.  In 3D, despite the Ma not scaling with Ra, we see similar density
inversions at high $\epsilon$ once Ra is sufficiently large.

\section{Discussion}
\refstepcounter{section}
\label{sec:discussion}
In this letter we have studied fundamental heat transport by 
stratified convection in simplified 2D and 3D polytropic atmospheres.
We argue that these atmospheres are the natural extension
of the RB problem to stratified systems, 
and are an ideal laboratory for understanding the basic 
properties of stratified convection. 
The primary difference we see between our 2D and 3D results is that the Ma
is not a function of Ra in 3D, and this has far-reaching consequences.  Otherwise,
we see only slightly different scaling in Nu(Ra) in 2D and 3D, 
which aligns with expectations
of Boussinesq theory and numerical simulation results at 
values of Pr $\geq$ 1 \cite{ahlers&all2009}.  

At low Ra and Ma, the scaling of Nu with Ra is reminiscent of RB convection.  
However, at high Ra and Ma, the scaling of Nu becomes weaker, changing
from a 1/3 to a 1/5 power law, a decrease by nearly a factor of 1/3.
Similarly, the Reynolds number switches from a 3/4 power law to
a 1/2 scaling in this regime shift, a similar decrease.
This seems to suggest that a large portion of the scaling of the Nusselt number
and Reynolds number in the low-Ma regime
comes from increases in velocity.  Once the velocity reaches
its natural maximum, heat transport can no longer benefit from increased velocities.
The same is true of the Reynolds number, which scales directly with system diffusivities
once the Mach number reaches its maximum value.

Time-dependent oscillating shear states have developed spontaneously
at values of $\epsilon=10^{-4}$ and Ra $\gtrsim 10^3$Ra$_{\text{crit}}$ in our 2D simulations.  
Similar states have been observed in 2D RB convection \cite{goluskin&all2014}, and these should
be studied in more detail.  These states are fundamentally in flux disequilibrium
and last for large time scales (O(100$t_b$)),
making it very difficult to choose a proper time window over which to calculate a 
meaningful Nusselt number.  It would
be beneficial for future studies to examine atmosperes with fixed flux or no slip
boundary conditions to better converge and study cases at low $\epsilon$ and high
Ra.

Finally, we have found that the stratification of 
these atmospheres evolves in a complex
manner.  Future work should aim to 
understand the importance of stratification on
convective heat transport, and the manner in which
boundary layer density inversions interact with
the formation and transport properties of thermal plumes.

Our studies
here will serve as a foundation both for understanding and 
comparing heat transport in stratified convection
to that in RB convection \cite{johnston&doering2009}, 
and for future studies of transport in stratified convection.  
These results can be used to determine if simplified equation sets, 
such as the anelastic equations, carry heat in the same manner as the 
FC equations.
This work will also be useful in coming to understand more realistic systems, 
such as rapidly rotating atmospheres \cite{julien&all2012},
atmospheres bounded by stable regions \cite{hurlburt&all1986}, 
or regions with realistic profiles of $\kappa$.



\subsection{acknowledgements}
EHA acknowledges the support of the University of Colorado's George 
Ellery Hale Graduate Student Fellowship.
This work was additionally supported by  NASA LWS grant number NNX16AC92G.  
Computations were conducted 
with support by the NASA High End Computing (HEC) Program through the NASA 
Advanced Supercomputing (NAS) Division at Ames Research Center on Pleiades
with allocations GID s1647 and GID g26133.
We thank Jon Aurnou, Axel Brandenburg, Keith Julien, Mark Rast, and Jeff Oishi 
for many useful discussions. We also thank the two anonymous referees whose
critical comments greatly improved the quality of this letter.

\bibliography{../biblio.bib}
\end{document}
