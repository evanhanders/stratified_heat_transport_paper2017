%\documentclass[iop]{emulateapj}
\documentclass[aps, prl, twocolumn, nofootinbib, groupedaddress, amsfonts, amssymb, amsmath]{revtex4-1}
\usepackage{graphicx}
\usepackage{bm}
\usepackage{natbib}
%\usepackage[colorlinks=True, linkcolor=blue, citecolor=blue]{hyperref}
%\usepackage[all]{hypcap}

\bibliographystyle{apsrev}

\newcommand{\Div}[1]{\ensuremath{\nabla\cdot\left( #1\right)}}
\newcommand{\angles}[1]{\ensuremath{\left\langle #1 \right\rangle}}
\newcommand{\grad}{\ensuremath{\nabla}}
\newcommand{\RB}{Rayleigh-B\'{e}nard }
\newcommand{\stressT}{\ensuremath{\bm{\bar{\bar{\Pi}}}}}
\newcommand{\lilstressT}{\ensuremath{\bm{\bar{\bar{\sigma}}}}}
\newcommand{\nrho}{\ensuremath{n_{\rho}}}
\newcommand{\approptoinn}[2]{\mathrel{\vcenter{
	\offinterlineskip\halign{\hfil$##$\cr
	#1\propto\cr\noalign{\kern2pt}#1\sim\cr\noalign{\kern-2pt}}}}}

\newcommand{\appropto}{\mathpalette\approptoinn\relax}

\newcommand\mnras{{MNRAS}}%
          % Monthly Notices of the RAS

\begin{document}
%%%%% Create nice title and abstract
\author{Evan H. Anders}
\affiliation{Department of Astrophysical \& Planetary Sciences, University of Colorado -- Boulder}
\affiliation{Laboratory for Atmospheric and Space Physics, Boulder, CO}
\author{Benjamin P. Brown}
\affiliation{Department of Astrophysical \& Planetary Sciences, University of Colorado -- Boulder}
\affiliation{Laboratory for Atmospheric and Space Physics, Boulder, CO}
\title{Convective heat transport in stratified atmospheres at low and high Mach number}

\begin{abstract}
%Convection in astrophysical systems is stratified and
%often occurs at high Rayleigh number (Ra) and low
%Mach number (Ma).
We study stratified convection in the context of 
plane-parallel, polytropically stratified atmospheres. 
We perform a suite of 2D and 3D simulations in which we vary the initial
superadiabaticity ($\epsilon$) and the Rayleigh number (Ra) while fixing the
initial density stratification, aspect
ratio, and Prandtl number.
The evolved heat transport, 
quantified by the Nusselt number (Nu),
follows scaling relationships similar to those found in the well-studied, 
incompressible Rayleigh-B\'{e}nard problem in both 2D and 3D and is not
appreciably affected by the magnitude of $\epsilon$.
The evolved Mach number (Ma) scales according to
$\text{Ma} \propto \epsilon^{1/2}\text{Ra}^{\alpha}$, where
$\alpha = 0$ in 3D and $\alpha \approx 1/4$ in 2D.
At large values of $\epsilon$, significant density inversions appear in the
evolved atmospheres and seem to persist in both 2D and 3D.
\end{abstract}
\maketitle


%%%%% Body of the paper
\section{Introduction}
\refstepcounter{section}
\label{sec:intro}
Convection transports energy in stellar and planetary atmospheres.
In these objects, flows are compressible and
feel the atmospheric stratification.  While in some systems this stratification 
is negligible, it is significant in regions such as
the convective envelope of the Sun, which spans 14 density scale heights.
In the bulk of these systems, especially in the deep interior far below the surface,
flows are at very low Mach number (Ma).  Unfortunately,
numerical constraints have restricted most studies of 
compressible convection to high Ma.
These prior studies \cite{graham1975, chan&all1982,
hurlburt&all1984, cattaneo&all1990, brummell&all1996,
brandenburg&all2005} have provided insight into the nature of
convection in the low temperature,
high Ma region near the Sun's surface. Few fundamental
properties of low Ma compressible convection, such as the scaling of
convective heat transport, are known.

In the widely-studied \RB (RB) problem of incompressible Boussinesq convection, 
a negative temperature gradient causes convective instability.
In the evolved solution, upflows and downflows are symmetrical, the
temperature in the interior becomes isothermal, and
the conductive flux ($\propto \grad T$) approaches 
zero there. 
For compressible convection in a stratified atmosphere, a
negative entropy gradient causes convective instability.
Early numerical experiments of moderate-to-high Ma compressible convection
in two \cite{graham1975, chan&all1982,
hurlburt&all1984, cattaneo&all1990} and three 
\cite{cattaneo&all1991, brandenburg&all2005, brummell&all1996} dimensions
revealed a different evolved state from the RB case.
Downflow lanes
become fast and narrow, and upflow lanes turn into broad, slow upwellings.
Furthermore, the \emph{entropy} gradient is negated by convection in the interior, so
a significant temperature gradient and conductive flux can persist despite
efficient convection.

In RB convection, there exist two primary dynamical control parameters: 
the Rayleigh number (Ra, the ratio of
buoyant driving to diffusive damping) and the Prandtl number 
(Pr, the ratio of viscous to thermal
diffusivity). These numbers control two useful
measures of turbulence in the evolved solution:
the Reynolds
number (Re, the strength of advection to viscous diffusion)
and the Peclet number (Pe, advection vs. thermal diffusion).  
In stratified atmospheres, the magnitude of the unstable entropy gradient
joins Ra and Pr as an important control parameter.  This 
\emph{superadiabatic excess} \cite{graham1975}, $\epsilon$,
sets the scale of the atmospheric entropy gradient.
We find here that $\epsilon$ primarily controls the Ma of the evolved solution.

In this letter we study the behavior of convective heat transport, 
quantified by the Nusselt number (Nu), in plane-parallel, 
two- and three-dimensional, polytropically stratified atmospheres.  
We vary $\epsilon$ and Ra while holding Pr, aspect ratio, boundary conditions,
and initial atmospheric stratification
constant.  We also examine the behavior of flow properties, as quantified by Ma and Re.
We find here that the scaling of Nu in stratified, compressible convection 
is similar to that in \RB convection,
and that this scaling is not appreciably changed by the magnitude of the superadiabaticity.

\section{Experiment} 
\refstepcounter{section}
\label{sec:experiment}
We examine a monatomic ideal gas with an adiabatic index of
$\gamma = 5/3$ whose equation of state is $P = R\rho T$. This is consistent with the approach used in earlier work 
\cite{graham1975, chan&all1982, brandenburg&all2005,
hurlburt&all1984, cattaneo&all1990, cattaneo&all1991, brummell&all1996} 
and is the simplest stratified extension of RB.
We study atmospheres which are initially polytropically stratified,
\begin{equation}
\begin{split}
\rho_0(z) &= \rho_{t}(1 + L_z - z)^m, \\
T_0(z)    &= T_{t}(1 + L_z - z),
\label{eqn:polytrope}
\end{split}
\end{equation}
where $m$ is the polytropic index and $L_z$ is the depth of the atmosphere.
The polytropic
index is set by the superadiabatic excess, $\epsilon = m_{ad} - m$, where
$m_{ad} = (\gamma - 1)^{-1}$ is the adiabatic value of $m$.
The height coordinate, $z$, increases upwards in the range $[0, L_z]$.
Subscript 0 indicates initial conditions and subscript $t$ indicates values
at $z = L_z$.   We
specify the depth of the atmosphere, $L_z = e^{n_{\rho}/m} - 1$, by choosing
the number of density scale heights, $n_{\rho}$, it spans initially.
Throughout this letter we set $n_{\rho} = 3$.    Satisfying hydrostatic
equilibrium sets the value of gravity, $g = RT_t (m + 1)$, which is
constant with depth.  We study atmospheres with aspect
ratios of 4 where both the $x$ and $y$ coordinates have the range $[0, 4L_z]$.
In our 2D cases, we only consider $x$ and $z$.

These domains are nondimensionalized by setting
all thermodynamic variables to unity at $z = L_z$, choosing
$R = T_t = \rho_t = 1$.  By this choice, the non-dimensional
length scale is the inverse temperature gradient scale and the 
timescale is the isothermal sound crossing time, 
$\tau_I$, of this unit length.
Meaningful convective dynamics occur on 
timescales of the atmospheric buoyancy time,
$t_b = \tau_I \sqrt{L_z/g\epsilon}$. 

At fixed $n_\rho$, convective dynamics are 
controlled by $\epsilon$ as well as the atmospheric diffusivities.
At a fixed value of
$\epsilon$, the diffusivities are set by the
Rayleigh number (Ra) and the Prandtl number (Pr),
\begin{equation}
\text{Ra}_{t} = \frac{g L_z^3 (\Delta S_0 / c_P)}{\nu_t\chi_t},
\qquad
\text{Pr} = \frac{\nu}{\chi},
\end{equation}
where $\Delta S_0 = \epsilon\ln (1 + L_z)$ is the initial entropy
difference between the top and bottom boundaries,
$c_P = \gamma/(\gamma-1)$ is the specific heat a
constant pressure, and $\nu$ and $\chi$ are the thermal diffusivity and
kinematic viscosity, respectively.
Throughout this work we specify
that Pr $= 1$ and is depth invariant.
Polytropes, as specified in (\ref{eqn:polytrope}),
are in thermal equilibrium. Thus, the initial thermal
conductivity, $\kappa_0 = \chi \rho_0$, is
constant with depth. 
By these
choices, $\nu(z) \equiv \chi(z) \equiv \chi_t / \rho_0$.
This formulation 
sets Ra at the bottom of the domain greater than
Ra$_t$ by a factor of $e^{2n_\rho}$. Henceforth
when we specify Ra we are referring to Ra$_t$.  
The full values of $\kappa = \rho\chi$ and 
$\mu = \rho\nu$ (the dynamic viscosity) are free to evolve as the density 
profile evolves.  

Under this formulation, diffusivities scale as
\begin{equation}
\chi_t = \nu_t \propto \sqrt{\frac{\epsilon}{\text{Ra}_t}}
\label{eqn:diffusivities}
\end{equation}
for constant Pr = 1, as in this work.
We carry out two experiments in this letter: one in which
we fix $\epsilon$ and increase Ra, thus increasing the diffusive
timescales, and a second in which we fix Ra and raise $\epsilon$,
thus modifying the ratio of the diffusive timescales to the
buoyant timescale.

We evolve the Fully Compressible Navier-Stokes equations,
\begin{align}
&\begin{aligned}
&\frac{\partial \ln\rho}{\partial t} + \grad\cdot\bm{u} 
    = -\bm{u}\cdot\grad\ln\rho,
	\label{eqn:continuity_eqn}
\end{aligned}\\
&\begin{aligned}
\frac{\partial\bm{u}}{\partial t} + \grad T - 
&\nu\grad\cdot\lilstressT - \lilstressT\cdot\grad\nu = \\
&-\bm{u}\cdot\grad\bm{u} - T\grad\ln\rho + \bm{g} + 
\nu\lilstressT\cdot\grad\ln\rho,
\label{eqn:momentum_eqn}
\end{aligned}\\
&\begin{aligned}
\frac{\partial T}{\partial t} -\frac{1}{c_V}\left(\right.\chi&\left.
    \grad^2 T + \grad T\cdot\grad\chi\right) = \\
	&-\bm{u}\cdot\grad T - (\gamma-1)T\grad\cdot{\bm{u}} \\
	&+ \frac{1}{c_V}\left(\chi\grad T \cdot\grad\ln\rho +
	\nu\left[\lilstressT\cdot\nabla\right]\cdot\bm{u}\right), 
	\label{eqn:energy_eqn}
\end{aligned}
\end{align}
with the viscous stress tensor given by
\begin{equation}
\sigma_{ij} \equiv \left(\frac{\partial u_i}{\partial x_j} + 
\frac{\partial u_j}{\partial x_i} - \frac{2}{3}\delta_{ij}\grad\cdot\bm{u}\right).
	\label{eqn:stress_tensor}
\end{equation}
Taking an inner product of
(\ref{eqn:momentum_eqn}) with $\rho\bm{u}$ and adding it to 
$\rho c_V\times$(\ref{eqn:energy_eqn}) reveals the full energy equation,
\begin{equation}
\frac{\partial}{\partial t}\left(\rho\left[\frac{|\bm{u}|^2}{2} + c_V T + \phi\right]\right) +
\Div{\bm{F}_{\text{conv}} + \bm{F}_{\text{cond}}} = 0,
	\label{eqn:energy_eqn_full}
\end{equation}
where
$
\bm{F}_{\text{conv}} \equiv \bm{F}_{\text{enth}} + \bm{F}_{\text{KE}} + \bm{F}_{\text{PE}} + \bm{F}_{\text{visc}}
$
is the convective flux and $\bm{F}_{\text{cond}} = -\kappa \grad T$
is the conductive flux.
The individual contributions to $\bm{F}_{\text{conv}}$ are the enthalpy flux, 
$\bm{F}_{\text{enth}} \equiv \rho\bm{u}(c_V T + P/\rho)$;
the kinetic energy flux, 
$\bm{F}_{\text{KE}} \equiv \rho|\bm{u}|^2\bm{u}/2$;
the potential energy flux,
$\bm{F}_{\text{PE}} \equiv \rho\bm{u}\phi$ (with $\phi \equiv -gz$);
and the viscous flux, 
$\bm{F}_{\text{visc}} \equiv -\rho\nu\bm{u}\cdot\lilstressT$.
Understanding how each of these fluxes interact  
is crucial in characterizing convective heat transport.

We utilize the 
Dedalus\footnote{http://dedalus-project.org/} \cite{burns&all2016} 
pseudospectral framework to time-evolve  
(\ref{eqn:continuity_eqn})-(\ref{eqn:energy_eqn}) 
using an implicit-explicit (IMEX), third-order, four-step 
Runge-Kutta timestepping scheme RK443 \cite{ascher&all1997}.  
We decompose our thermodynamic variables such that $T = T_0 + T_1$ and
$\ln\rho = (\ln\rho)_0 + (\ln\rho)_1$, 
and the velocity is $\bm{u} = u\hat{x} + v\hat{y} +  w\hat{z}$.
In our 2D runs, $v = 0$.
Subscript 0 variables, set by (\ref{eqn:polytrope}), 
have no time derivative and vary only in $z$.
Variables are time-evolved on a dealiased Chebyshev (vertical)
and Fourier (horizontal, periodic) domain in which the
physical grid dimensions are 3/2 the size of the coefficient grid.  
Domain sizes range from
64x256 coefficients at the lowest values of 
Ra to 1024x4096 coefficients at Ra $> 10^{7}$ in 2D,
and from 64x128$^2$ to  256x512$^2$ in 3D. 
By using IMEX timestepping, we implicitly step the 
stiff linear acoustic wave contribution and are able to
efficiently study flows at high ($\sim 1$) 
and low ($\sim 10^{-4}$) Ma.  Our equations take the form
of the FC equations in \cite{lecoanet&all2014}, extended to include
$\nu$ and $\chi$ which vary with depth, and we follow the approach there.
This IMEX approach has been successfully 
tested against a nonlinear benchmark  of the compressible 
Kelvin-Helmholtz instability \cite{Lecoanet_et_al_2016_KH}.

We impose impenetrable, stress free, fixed temperature boundary conditions at
the top and bottom of the domain such that 
$w = \partial_z u = T_1 = 0$ at $z = \{0, L_z\}$. 
$T_1$ is initially filled with
random white noise whose magnitude is infinitesimal
compared to $T_0$ and $\epsilon$.
We filter this noise spectrum in coefficient space, 
such that only the lower 25\% of the coefficients
have power. All reported results are taken from time averages
over many $t_b$ beginning \{100, 40\}$t_b$
after the start of our \{2D, 3D\} simulations in order to
assure our results are not biased by the convective transient.

\section{Results \& Discussion}
\refstepcounter{section}
\label{sec:results}


\begin{figure}[t]
\includegraphics[width=3.4375in]{./figs/ma_v_Ra.png}
\caption{The mean adiabatic Mach number of long-time-averaged profiles
is shown.  Error bars show the full range of Ma over the depth of the
atmosphere.
(a) Ma, at various values of Ra, 
is plotted as a function of $\epsilon$.
(b) Ma, at various values of $\epsilon$, 
is plotted as a function of Ra/Ra$_{\text{crit}}$.
Larger symbols with white dots designate 3D runs.
\label{fig:ma_v_eps} }
\end{figure}


Solutions were time-evolved until a long time average of the fluxes
showed little
variance with depth. A linear stability analysis determined
that convective onset
occurs at $\text{Ra}_{\text{crit}} = \{11.15, 10.06, 10.97, 10.97\}$ 
for $\epsilon = \{1.0, 0.5, 10^{-4}, 10^{-7}\}$, respectively.  

We measure the adiabatic Mach number (Ma = $|\bm{u}|/\sqrt{\gamma T}$),
and find that it is a strong function of 
$\epsilon$ and a weak function of Ra.  
In our 2D runs, when Ma $< 1$, we observe a scaling law of
Ma(Ra$, \epsilon) \appropto \epsilon^{1/2}$Ra$^{1/4}$.
This relation breaks down as the mean
Ma approaches 1 (see Fig. \ref{fig:ma_v_eps}).  This transition
occurs near Ra/Ra$_\text{crit} \approx \{10^{2}, 10^{3}\}$ for $\epsilon = \{1, 0.5\}$.
We conjecture that the scaling of Ma with Ra is due to high-velocity ``spinners'' which
form between upflow and downflow lanes in 2D, and which are constantly fed by 
the relatively stationary upflows and downflows (BEN NEED CITES).
In our limited 3D runs, Ma appears to be a function of $\epsilon$ alone, with
Ma $\appropto \epsilon^{1/2}$, so at high Ra, Ma$_{\text{3D}} < \text{Ma}_{\text{2D}}$.
Simulations in the range of Ra/Ra$_{\text{crit}} > 10^3$ at $\epsilon = 10^{-4}$
exhibited ``windy'' states of convection, in which a large-scale horizontal
shearing flow replaced the more standard upflow/downflow morphology of
convection, similar to analogous states which have been studied in
RB convection \cite{goluskin&all2014}.  These states are represented in Figs. 
\ref{fig:ma_v_eps}, \ref{fig:re_and_nu_v_ra}, \& \ref{fig:nrho_v_ra}
as hatched points, and while this phenomeonon does not appear to greatly modify the
scaling of fluid properties measured in this work, these states warrant
further investigation.

\begin{figure}[t]
\includegraphics[width=3.4375in]{./figs/snapshots_fig.png}
\caption{Characteristic entropy fluctuations in evolved 2D flows roughly
140$t_b$ after the start of simulations. 
The time- and horizontally-averaged profile is removed in all cases.  
(a) A low Ma flow at moderate Ra. (b) A high Ma flow at the same Ra as in (a).
(c) A high Ma flow at high Ra.
Shock systems can be seen in the upper atmosphere of the high Ma flows,
for example at $(x, z) \sim (70, 15-19)$ in (b) and $(x, z) \sim 
(65, 17-19)$ in (c).
\label{fig:entropy_snapshots} }
\end{figure}

Low Ma flows (e.g., $\epsilon = 10^{-4}$)
display the classic narrow downflow and broad upflow lanes of stratified
convection (Fig. \ref{fig:entropy_snapshots}a).
At high Ma (e.g., $\epsilon = 0.5$, Ra/Ra$_{\text{crit}} \gtrsim 10^3$), 
bulk thermodynamic structures are similar but
shock systems form in the upper atmosphere near downflow lanes 
(Fig. \ref{fig:entropy_snapshots}b\&c).
These shock phenomena were reported in
two \cite{cattaneo&all1990} and 
three \cite{malagoli&all1990} dimensional polytropic simulations previously.
As in (\ref{eqn:diffusivities}), diffusivities become small as Ra 
is increased to large values (Fig. \ref{fig:entropy_snapshots}c) and
thermodynamic structures break up into small eddies which traverse the
domain repeatedly before diffusing.
Furthermore, the value of $\epsilon$ sets the size of the
evolved thermodynamic fluctuations from their adiabatic values,
as evidenced by the scalings of the colorbars.
%These fluctuations are very small when $\epsilon$ and Ma are small,
%but can be O(1) for values of $\epsilon$ near 1.

%Our choice of fixed temperature, stress free boundary conditions 
%allows the flux at the boundaries to vary, while also allowing
%for mean horizontal velocity profiles along the boundaries (winds).
%We find these windy convective states, reminiscent of similar states
%in RB convection \cite{goluskin&all2014}, at $\epsilon = 10^{-4}$ and
%Ra/Ra$_{\text{crit}} \gtrsim 10^{3}$. Roll states (Fig. \ref{fig:entropy_snapshots}a) with high convective
%transport alternate with windy states of lower transport.


\begin{figure}[t]
\includegraphics[width=3.4375in]{./figs/re_and_nu_v_Ra.png}
\caption{
Flow properties at high and low $\epsilon$. 
(a) Nu vs. Ra/Ra$_{\text{crit}}$.
Errors bars indicate the variance of Nu with depth;
large error bars indicate a poorly converged solution.
(b) Re vs. Ra/Ra$_{\text{crit}}$.
Re is measured at the midplane of the atmosphere.
Larger symbols with white dots designate 3D runs.
 \label{fig:re_and_nu_v_ra}
}
\end{figure}

The efficiency of convection is quantified by the Nusselt number (Nu).  
Nu is well-defined in RB convection
as the total flux normalized by the steady-state conductive flux 
\cite{johnston&doering2009, otero&all2002}.
In stratified convection Nu is more difficult to define, and we use
a modified version of a traditional stratified Nusselt number 
\cite{graham1975,hurlburt&all1984},
\begin{equation}
\text{Nu} \equiv \frac{\langle F_{\text{conv,z}} + F_{\text{cond,z}} - F_{\text{A}}\rangle}
{\langle F_{\text{cond,z}} - F_{\text{A}}\rangle} 
= 1 + \frac{\langle F_{\text{conv,z}}\rangle}{\langle F_{\text{cond,z}} - F_{\text{A}} \rangle}
\label{eqn:nusselt}
\end{equation}
where $F_{\text{conv,z}}$ and $F_{\text{cond,z}}$ are the 
z-components of $\bm{F}_{\text{conv}}$ and $\bm{F}_{\text{cond}}$,
and $\langle \rangle$ are volume averages.  
$F_{\text{A}} \equiv -\langle\kappa\rangle \partial_z T_{\text{ad}}$ 
is the conductive flux of the proper corresponding adiabatic atmosphere.
For a compressible, ideal gas in hydrostatic equilibrium,
$\partial_z T_{\text{ad}} \equiv - g / c_{P}$ \cite{spiegel&veronis1960}.  
It is important to measure the evolved value of
$\langle \kappa \rangle = \langle \rho\chi \rangle$, which is nearly
$\kappa_0$ for small $\epsilon$ but changes appreciably for large
values of $\epsilon$.
In incompressible Boussinesq convection, where $\grad S = 0$ only when 
$\grad T = 0$, this definition reduces to the traditionally defined
Nusselt number \cite{otero&all2002, johnston&doering2009}.

The variation of Nu with Ra is shown in 
Fig. \ref{fig:re_and_nu_v_ra}a.  We find that the Nu depends primarily
on Ra, not on $\epsilon$, except where dynamical regimes change.
In 2D and at low to moderate Ra, 
Nu $\appropto$ Ra$^{1/3}$ regardless of $\epsilon$,
reminiscent of scaling laws in \RB boundary layer theory 
\cite{grossman&lohse2000, ahlers&all2009, king&all2012}.
As the flow becomes supersonic,  Nu $\appropto$ Ra$^{1/5}$.
%2D runs at $\text{Ra }\gtrsim 10^3$Ra$_{\text{crit}}$
%and $\epsilon = 10^{-4}$ exhibit windy convective states.
%In these runs, the system is in flux equilibrium only when
%time averaged over roll states and windy states.
%A long time average
%over both of these states produces a flat flux profile which
%can be sensibly analyzed, but the presence of these time-dependent states
%makes it difficult to equilibrate these higher Ra cases.
It is also important to note that
the value of Nu is heavily dependent upon the specific thermodynamic
structures of the solution, as double roll states will transport
heat more efficiently than single roll states, and slight changes in
Ra can result in a simulation latching on to one solution or the other. 
Select simulations were run at aspect ratios of 8 and 16, and identical flow
morphologies were obtained, which suggests that these states are sensitive to
parameters other than the width of the domain.
In our limited 3D runs, it appears that Nu $\appropto$ Ra$^{2/7}$, a classic scaling law
seen in RB studies \cite{johnston&doering2009}.

The rms Reynolds number (Re = $|\bm{u}|L_z/\nu$) and Peclet number
(Pe = Pr Re)
compare the importance of advection to diffusion in the evolved
convective state.  For Pr = 1, such as in this work, Pe = Re.  
Our choice of $\{\nu,\chi\}\propto \rho_0^{-1}$ drastically changes
the value of Re between the top and bottom of the atmosphere.  We report values of
Re at the midplane ($z=L_z/2$) of the atmosphere in
Fig. \ref{fig:re_and_nu_v_ra}b.  Largely we find that Re
depends on Ra, but not $\epsilon$, except when the flow states
change.
In 2D and at low Ra, Re $\appropto$ Ra$^{3/4}$.  The heightened scaling
of Re in 2D is due to the scaling of velocity (Ma) with Ra, as is
seen in Fig. \ref{fig:ma_v_eps}.  When the flows
become supersonic, 
this scaling gives way to Re $\appropto$ Ra$^{1/2}$, as expected
by the scaling of diffusivities in (\ref{eqn:diffusivities}).
In our limited 3D runs, the scaling of Re is
Re $\appropto$ Ra$^{1/2}$, consistent with the supersonic results.

\begin{figure}[t]
\includegraphics[width=3.4375in]{./figs/density_v_ra.png}
\caption{\label{fig:nrho_v_ra} 
Solid symbols show the density contrast measured
in density scale heights between the upper and lower boundary, 
$n_\rho = \ln[\rho(z=0)/\rho(z=L_z)]$.
Empty symbols show 
$n_\rho = \ln[\text{max}(\rho)/\text{min}(\rho)]$. 
At low $\epsilon$ the evolved
$n_{\rho}$ is close to the initial conditions of $n_\rho = 3$.  
At high $\epsilon$,
the density stratification decreases.  Once the mean 
Ma approaches 1 (at Ra/Ra$_{\text{crit}} \approx \{10^2, 10^3\}$ for $\epsilon = \{1, 0.5\}$
as in Fig. \ref{fig:ma_v_eps}b), density inversions form within the thermal
boundary layers. Larger symbols represent 3D runs.}
\end{figure}

In the evolved state, the flows change the density stratification,
as shown in Fig. \ref{fig:nrho_v_ra}.
Here we take two different measurements of the stratification present in the
time-averaged, horizontally-averaged density profile: the density difference
between the max and min value of the density profile (empty symbols) and the
density difference between the top and bottom of the atmosphere (solid symbols).
We find that supersonic flows support persistent density inversions (empty symbols)
in the boundary layers, as
was reported by \cite{brandenburg&all2005}.  We find this in 2D and 3D, even
at very large $\epsilon$.
Surprisingly, the evolved $n_\rho$ is always less than the initial $n_\rho = 3$,
and turbulent pressure support plays a larger role than atmospheric slumping.
The agreement of Nu \& Re across $\epsilon$ (Fig. \ref{fig:re_and_nu_v_ra}), 
particularly at low Ra in which all four of our cases collapse onto a single
power law, is striking in light of the vastly different evolved stratifications
felt by the flows. 

%Discussion

In summary, we have found that the evolved flow properties of stratified,
compressible convection scale in a manner reminiscent of RB convection.
We argue that polytropically stratified atmospheres are the natural
extension of the RB problem with an additional control parameter, $\epsilon$,
whose primary role is to set the Ma of the flows.  We show that other properties
of the evolved solutions (Nu, Re) are nearly identical at vastly different values
of $\epsilon$, except for where there is a transition between the subsonic
and supersonic regimes.  We also see that Nu scales similarly in 3D and 2D,
and that Ma in 3D solutions seems to be a function of $\epsilon$ alone,
allowing for simple specification of the evolved Ma using input parameters.

%We observe interesting phenomena, such as time-dependent, windy
%states of shear in our 2D simulations.  
%Unlike in RB \cite{goluskin&all2014}, our solutions with
%windy states have transport which scales in line with our non-windy solutions.
The stratification of 
these polytropic atmospheres evolves in a complex
manner.  Future work should aim to 
understand the importance of stratification on
convective heat transport and other flow properties.

Our studies here will serve as a foundation for 
comparing heat transport in stratified convection
to that in RB convection \cite{johnston&doering2009}
and for better quantifying transport in stratified convection.  
These results can be used to determine if fluid properties
scale appropriately in simplified equation sets, 
such as the anelastic equations.
This work will also be useful in coming to understand more realistic systems, 
such as rapidly rotating atmospheres \cite{julien&all2012},
atmospheres bounded by stable regions \cite{hurlburt&all1986}, 
and regions with realistic profiles of $\kappa$.



\subsection{acknowledgements}
EHA acknowledges the support of the University of Colorado's George 
Ellery Hale Graduate Student Fellowship.
This work was additionally supported by  NASA LWS grant number NNX16AC92G.  
Computations were conducted 
with support by the NASA High End Computing (HEC) Program through the NASA 
Advanced Supercomputing (NAS) Division at Ames Research Center on Pleiades
with allocations GID s1647 and GID g26133.
We thank Jon Aurnou, Axel Brandenburg, Keith Julien, Mark Rast, and Jeff Oishi 
for many useful discussions. We also thank the two anonymous referees whose
careful comments greatly improved the quality of this letter.

\bibliography{../biblio.bib}
\end{document}
